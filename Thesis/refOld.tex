@article{Smith2017,
abstract = {We report the observation of a spin-flip process in a quantum dot whereby a dark exciton with total angular momentum L = 2 becomes a bright exciton with L = 1. The spin-flip process is revealed in the decay dynamics following nongeminate excitation. We are able to control the spin-flip rate by more than an order of magnitude simply with a dc voltage. The spin-flip mechanism involves a spin exchange with the Fermi sea in the back contact of our device and corresponds to the high temperature Kondo regime. We use the Anderson Hamiltonian to calculate a spin-flip rate, and we find excellent agreement with the experimental results.},
archivePrefix = {arXiv},
arxivId = {0812.2957},
author = {Smith, J. M. and Dalgarno, P. A. and Warburton, R. J. and Govorov, A. O. and Karrai, K. and Gerardot, B. D. and Petroff, P. M. and Mi, X. and Cady, J. V. and Zajac, D. M. and Deelman, P. W. and Petta, J. R. and Qassemi, F. and Coish, W. A. and Wilhelm, F. K.},
doi = {10.1126/science.aal2469},
eprint = {0812.2957},
file = {:Q$\backslash$:/spin-QED/Steffi Stuff/References/OtherQubits/QuassemiPhysRevLett.102.176806 Stationary and Transient Leakage Current in the Pauli Spin Blockade.pdf:pdf;:Q$\backslash$:/spin-QED/Steffi Stuff/References/OtherQubits/science.aal2469.full{\_}cavcitycoupling.pdf:pdf;:Q$\backslash$:/spin-QED/Steffi Stuff/References/OtherQubits/smith05PRL(cotunneling optical QD).pdf:pdf},
isbn = {1095-9203 (Electronic)$\backslash$r0036-8075 (Linking)},
issn = {00319007},
journal = {Physical Review Letters},
number = {17},
pages = {1--4},
pmid = {16090209},
title = {{Voltage control of the spin dynamics of an exciton in a semiconductor quantum dot}},
volume = {102},
year = {2017}
}
@article{Neder2014,
archivePrefix = {arXiv},
arxivId = {arXiv:0801.3613},
author = {Neder, Izhar and Rudner, Mark S. and Halperin, Bertrand I.},
doi = {10.1103/PhysRevB.89.085403},
eprint = {arXiv:0801.3613},
issn = {1098-0121},
journal = {Physical Review B},
number = {8},
pages = {085403},
title = {{Theory of coherent dynamic nuclear polarization in quantum dots}},
url = {http://link.aps.org/doi/10.1103/PhysRevB.89.085403},
volume = {89},
year = {2014}
}
@article{Nowack2008,
author = {Nowack, K C},
doi = {10.1126/science.1148092},
file = {:C$\backslash$:/Users/Steffi/AppData/Local/Mendeley Ltd./Mendeley Desktop/Downloaded/Nowack - 2008 - Coherent Control of a Single Electron.pdf:pdf},
number = {2007},
title = {{Coherent Control of a Single Electron}},
volume = {1430},
year = {2008}
}
@article{Brataas2011,
abstract = {We consider nuclear spin dynamics in a two-electron double dot system near the intersection of the electron spin singlet {\$}S{\$} and the lower energy component {\$}T{\_}{\{}+{\}}{\$} of the spin triplet. The electron spin interacts with nuclear spins and is influenced by the spin-orbit coupling. Our approach is based on a quantum description of the electron spin in combination with the coherent semiclassical dynamics of nuclear spins. We consider single and double Landau-Zener passages across the {\$}S{\$}-{\$}T{\_}{\{}+{\}}{\$} anticrossings. For linear sweeps, the electron dynamics is expressed in terms of parabolic cylinder functions. The dynamical nuclear polarization is described by two complex conjugate functions {\$}\backslashLambda {\^{}}{\{}\backslashpm{\}}{\$} related to the integrals of the products of the singlet and triplet amplitudes {\$}{\{}\backslashtilde{\{}c{\}}{\}}{\_}{\{}S{\}}{\^{}}{\{}\backslashast{\}}{\{}\backslashtilde{\{}c{\}}{\}}{\_}{\{}T{\_}{\{}+{\}}{\}}{\$} along the sweep. The real part {\$}P{\$} of {\$}\backslashLambda {\^{}}{\{}\backslashpm{\}}{\$} is related to the {\$}S{\$}-{\$}T{\_}{\{}+{\}}{\$} spin-transition probability, accumulates in the vicinity of the anticrossing, and for long linear passages coincides with the Landau-Zener probability {\$}P{\_}{\{}LZ{\}}=1-e{\^{}}{\{}-2\backslashpi \backslashgamma{\}}{\$}, where {\$}\backslashgamma {\$} is the Landau-Zener parameter. The imaginary part {\$}Q{\$} of {\$}\backslashLambda{\^{}}{\{}+{\}}{\$} is specific for the nuclear spin dynamics, accumulates during the whole sweep, and for {\$}\backslashgamma \backslashgtrsim 1{\$} is typically an order of magnitude larger than {\$}P{\$}. {\$}Q{\$} has a profound effect on the nuclear spin dynamics, by (i) causing intensive shake-up processes among the nuclear spins and (ii) producing a high nuclear spin generation rate when the hyperfine and spin-orbit interactions are comparable in magnitude. We find analytical expressions for the back-action of the nuclear reservoir represented via the change in the Overhauser fields the electron subsystem experiences.},
archivePrefix = {arXiv},
arxivId = {1104.4591},
author = {Brataas, Arne and Rashba, Emmanuel I.},
doi = {10.1103/PhysRevB.84.045301},
eprint = {1104.4591},
file = {:C$\backslash$:/Users/Steffi/AppData/Local/Mendeley Ltd./Mendeley Desktop/Downloaded/Brataas, Rashba - 2011 - Nuclear dynamics during Landau-Zener singlet-triplet transitions in double quantum dots.pdf:pdf},
issn = {10980121},
journal = {Physical Review B - Condensed Matter and Materials Physics},
number = {4},
pages = {1--17},
title = {{Nuclear dynamics during Landau-Zener singlet-triplet transitions in double quantum dots}},
volume = {84},
year = {2011}
}
@article{Laucht2014,
abstract = {A single atom is the prototypical quantum system, and a natural candidate for a quantum bit, or qubit--the elementary unit of a quantum computer. Atoms have been successfully used to store and process quantum information in electromagnetic traps, as well as in diamond through the use of the nitrogen-vacancy-centre point defect. Solid-state electrical devices possess great potential to scale up such demonstrations from few-qubit control to larger-scale quantum processors. Coherent control of spin qubits has been achieved in lithographically defined double quantum dots in both GaAs (refs 3-5) and Si (ref. 6). However, it is a formidable challenge to combine the electrical measurement capabilities of engineered nanostructures with the benefits inherent in atomic spin qubits. Here we demonstrate the coherent manipulation of an individual electron spin qubit bound to a phosphorus donor atom in natural silicon, measured electrically via single-shot read-out. We use electron spin resonance to drive Rabi oscillations, and a Hahn echo pulse sequence reveals a spin coherence time exceeding 200 µs. This time should be even longer in isotopically enriched (28)Si samples. Combined with a device architecture that is compatible with modern integrated circuit technology, the electron spin of a single phosphorus atom in silicon should be an excellent platform on which to build a scalable quantum computer.},
archivePrefix = {arXiv},
arxivId = {1305.4481},
author = {Veldhorst, M. and Yang, Changyi H. and Hwang, J. C. C. and Huang, W. and Dehollain, Juan P. and Muhonen, Juha T. and Simmons, S. and Laucht, Arne and Hudson, Fay E. and Itoh, Kohei M. and Morello, Andrea and Dzurak, Andrew S. and Tosi, Guilherme and Mohiyaddin, Fahd a. and Huebl, Hans and Morello, Andrea and Pla, Jarryd James and Mohiyaddin, Fahd a. and Tan, Kuan Y. and Dehollain, Juan P. and Rahman, Rajib and Klimeck, Gerhard and Jamieson, David N. and Dzurak, Andrew S. and Morello, Andrea and Lim, Wee H. and Morton, John J L and Jamieson, David N. and Dzurak, Andrew S. and Morello, Andrea and Muhonen, Juha T. and Laucht, Arne and Simmons, S. and Dehollain, Juan P. and Kalra, Rachpon and Hudson, Fay E. and Freer, S. and Itoh, Kohei M. and Jamieson, David N. and McCallum, Jeffrey C. and Dzurak, Andrew S. and Morello, Andrea and Laucht, Arne and Hudson, Fay E. and Kalra, Rachpon and Sekiguchi, Takeharu and Itoh, Kohei M. and Jamieson, David N. and McCallum, Jeffrey C. and Dzurak, Andrew S. and Morello, Andrea and Pla, Jarryd James and Zwanenburg, Floris A. and Chan, Kok W. and Tan, Kuan Y. and Huebl, Hans and M{\"{o}}tt{\"{o}}nen, Mikko and Nugroho, Christopher D. and Yang, Changyi H. and van Donkelaar, Jessica A. and Alves, Andrew D.C. C and Jamieson, David N. and Escott, Christopher C. and Hollenberg, Lloyd C.L. L and Clark, Robert G. and Dzurak, Andrew S. and Huebl, Hans and {Willems Van Beveren}, L. H. and Hollenberg, Lloyd C.L. L and Jamieson, David N. and Dzurak, Andrew S. and Clark, Robert G. and Laucht, Arne and Muhonen, Juha T. and Mohiyaddin, Fahd a. and Kalra, Rachpon and Dehollain, Juan P. and Freer, S. and Hudson, Fay E. and Veldhorst, M. and Rahman, Rajib and Klimeck, Gerhard and Itoh, Kohei M. and Jamieson, David N. and McCallum, Jeffrey C. and Dzurak, Andrew S. and Morello, Andrea and Laucht, Arne and Hill, Charles D. and Morello, Andrea and Pablo, Juan and Lorenzana, Dehollain and Pla, Jarryd James and Dehollain, Juan P. and Muhonen, Juha T. and Tan, Kuan Y. and Saraiva, Andre and Jamieson, David N. and Dzurak, Andrew S. and Morello, Andrea and Dehollain, Juan P. and Laucht, Arne and Hudson, Fay E. and Sekiguchi, Takeharu and Itoh, Kohei M. and Jamieson, David N. and McCallum, Jeffrey C. and Dzurak, Andrew S. and Morello, Andrea and Pla, Jarryd James and Tan, Kuan Y. and Dehollain, Juan P. and Lim, Wee H. and Morton, John J L and Zwanenburg, Floris A. and Jamieson, David N. and Dzurak, Andrew S. and Morello, Andrea and Pla, Jarryd James and Zwanenburg, Floris A. and Chan, Kok W. and Tan, Kuan Y. and Huebl, Hans and M{\"{o}}tt{\"{o}}nen, Mikko and Nugroho, Christopher D. and Yang, Changyi H. and van Donkelaar, Jessica A. and Alves, Andrew D.C. C and Jamieson, David N. and Escott, Christopher C. and Hollenberg, Lloyd C.L. L and Clark, Robert G. and Dzurak, Andrew S. and Laucht, Arne and Kalra, Rachpon and Muhonen, Juha T. and Dehollain, Juan P. and Mohiyaddin, Fahd a. and Hudson, Fay E. and McCallum, Jeffrey C. and Jamieson, David N. and Dzurak, Andrew S. and Morello, Andrea},
doi = {10.1038/nature11449},
eprint = {1305.4481},
file = {:Q$\backslash$:/spin-QED/Steffi Stuff/References/SiQubits/Morello/veldhorst2015{\_}nature15263.pdf:pdf;:Q$\backslash$:/spin-QED/Steffi Stuff/References/SiQubits/Morello/Muhonen2015{\_}RB.pdf:pdf;:Q$\backslash$:/spin-QED/Steffi Stuff/References/SiQubits/Morello/Morello2010{\_}nature09392{\_}supplementary.pdf:pdf;:Q$\backslash$:/spin-QED/Steffi Stuff/References/SiQubits/Morello/Muhonen2014{\_}nnano.2014.211{\_}supplement.pdf:pdf;:Q$\backslash$:/spin-QED/Steffi Stuff/References/SiQubits/Morello/Pla2012{\_}nature11449.pdf:pdf;:Q$\backslash$:/spin-QED/Steffi Stuff/References/SiQubits/Morello/Pla2012{\_}nature11449{\_}supp.pdf:pdf;:Q$\backslash$:/spin-QED/Steffi Stuff/References/SiQubits/Morello/Veldhorst2015{\_}Exchange1411.5760v1.pdf:pdf;:Q$\backslash$:/spin-QED/Steffi Stuff/References/SiQubits/Morello/Pla2014{\_}PhysRevLett.113.246801.pdf:pdf;:Q$\backslash$:/spin-QED/Steffi Stuff/References/SiQubits/Morello/SpinQED{\_}AIPadvances{\_}v2{\_}preprint.pdf:pdf;:Q$\backslash$:/spin-QED/Steffi Stuff/References/SiQubits/Morello/Morello2009{\_}PhysRevB.80.081307.pdf:pdf;:Q$\backslash$:/spin-QED/Steffi Stuff/References/SiQubits/Morello/Laucht2015{\_}ElectricalControl.pdf:pdf;:Q$\backslash$:/spin-QED/Steffi Stuff/References/SiQubits/Morello/Pla2013{\_}nature12011.pdf:pdf;:Q$\backslash$:/spin-QED/Steffi Stuff/References/SiQubits/Morello/Laucht2014{\_}1.4867905.pdf:pdf;:Q$\backslash$:/spin-QED/Steffi Stuff/References/SiQubits/Morello/Morello2010{\_}nature09392.pdf:pdf;:Q$\backslash$:/spin-QED/Steffi Stuff/References/SiQubits/Morello/Muhonen2014{\_}nnano.2014.211.pdf:pdf;:Q$\backslash$:/spin-QED/Steffi Stuff/References/SiQubits/Morello/Dehollain2014{\_}PhysRevLett.112.236801.pdf:pdf;:Q$\backslash$:/spin-QED/Steffi Stuff/References/SiQubits/Morello/Kalra2014{\_}PhysRevX.4.021044.pdf:pdf;:Q$\backslash$:/spin-QED/Steffi Stuff/References/SiQubits/Morello/Jarryd Pla (1-qubit NatSi).pdf:pdf;:Q$\backslash$:/spin-QED/Steffi Stuff/References/SiQubits/Morello/Juan Pablo Dehollain (1-qubit 28Si).pdf:pdf},
isbn = {0028-0836},
issn = {00280836},
journal = {Nature},
keywords = {,Nanophysics,Quantum information,Quantum physics,quantum information,silicon,single dopant},
number = {7316},
pages = {1--5},
pmid = {25305745},
publisher = {Nature Publishing Group},
title = {{A single-atom electron spin qubit in silicon}},
url = {http://www.ncbi.nlm.nih.gov/pubmed/23598342 http://arxiv.org/abs/1402.7140 http://advances.sciencemag.org/cgi/doi/10.1126/sciadv.1500022 http://dx.doi.org/10.1088/0953-8984/27/15/154205 http://dx.doi.org/10.1038/nature11449 http://dx.doi.org/10.1063/1.4893242{\%}5Cnhttp://scitation.aip.org/content/aip/journal/adva/4/8?ver=pdfcov http://arxiv.org/abs/1411.5760{\%}0Ahttp://dx.doi.org/10.1038/nature15263},
volume = {9},
year = {2014}
}
@book{GardinerC.W.Zoller2004,
author = {Gardiner, C. W. and Zoller, P.},
edition = {3},
file = {:C$\backslash$:/Users/Steffi/AppData/Local/Mendeley Ltd./Mendeley Desktop/Downloaded/Gardiner, Zoller - 2004 - Quantum Noise A Handbook of Markovian and Non-Markovian Quantum Stochastic Methods with Applications to Quantu.pdf:pdf},
publisher = {Springer-Verlag},
title = {{Quantum Noise: A Handbook of Markovian and Non-Markovian Quantum Stochastic Methods with Applications to Quantum Optics}},
year = {2004}
}
@article{Chekhovich2013,
abstract = {The interaction of an electronic spin with its nuclear environment, an issue known as the central spin problem, has been the subject of considerable attention due to its relevance for spin-based quantum computation using semiconductor quantum dots. Independent control of the nuclear spin bath using nuclear magnetic resonance techniques and dynamic nuclear polarization using the central spin itself offer unique possibilities for manipulating the nuclear bath with significant consequences for the coherence and controlled manipulation of the central spin. Here we review some of the recent optical and transport experiments that have explored this central spin problem using semiconductor quantum dots. We focus on the interaction between 10(4)-10(6) nuclear spins and a spin of a single electron or valence-band hole. We also review the experimental techniques as well as the key theoretical ideas and the implications for quantum information science.},
author = {Chekhovich, E a and Makhonin, M N and Tartakovskii, a I and Yacoby, a and Bluhm, H and Nowack, K C and Vandersypen, L M K},
doi = {10.1038/nmat3652},
file = {:C$\backslash$:/Users/Steffi/AppData/Local/Mendeley Ltd./Mendeley Desktop/Downloaded/Chekhovich et al. - 2013 - Nuclear spin effects in semiconductor quantum dots.pdf:pdf},
isbn = {1476-1122},
issn = {1476-1122},
journal = {Nature materials},
number = {6},
pages = {494--504},
pmid = {23695746},
publisher = {Nature Publishing Group},
title = {{Nuclear spin effects in semiconductor quantum dots.}},
url = {http://www.ncbi.nlm.nih.gov/pubmed/23695746},
volume = {12},
year = {2013}
}
@article{Petersson2012,
abstract = {Electron spins trapped in quantum dots have been proposed as basic building blocks of a future quantum processor. Although fast, 180-picosecond, two-quantum-bit (two-qubit) operations can be realized using nearest-neighbour exchange coupling, a scalable, spin-based quantum computing architecture will almost certainly require long-range qubit interactions. Circuit quantum electrodynamics (cQED) allows spatially separated superconducting qubits to interact via a superconducting microwave cavity that acts as a 'quantum bus', making possible two-qubit entanglement and the implementation of simple quantum algorithms. Here we combine the cQED architecture with spin qubits by coupling an indium arsenide nanowire double quantum dot to a superconducting cavity. The architecture allows us to achieve a charge-cavity coupling rate of about 30 megahertz, consistent with coupling rates obtained in gallium arsenide quantum dots. Furthermore, the strong spin-orbit interaction of indium arsenide allows us to drive spin rotations electrically with a local gate electrode, and the charge-cavity interaction provides a measurement of the resulting spin dynamics. Our results demonstrate how the cQED architecture can be used as a sensitive probe of single-spin physics and that a spin-cavity coupling rate of about one megahertz is feasible, presenting the possibility of long-range spin coupling via superconducting microwave cavities.},
archivePrefix = {arXiv},
arxivId = {arXiv:1205.6767v1},
author = {Petersson, K. D. and McFaul, L. W. and Schroer, M. D. and Jung, M. and Taylor, J. M. and Houck, a. a. and Petta, J. R.},
doi = {10.1038/nature11559},
eprint = {arXiv:1205.6767v1},
file = {:C$\backslash$:/Users/Steffi/AppData/Local/Mendeley Ltd./Mendeley Desktop/Downloaded/Petersson et al. - 2012 - Circuit quantum electrodynamics with a spin qubit.pdf:pdf},
isbn = {1476-4687 (Electronic)$\backslash$r0028-0836 (Linking)},
issn = {0028-0836},
journal = {Nature},
number = {7420},
pages = {380--383},
pmid = {23075988},
publisher = {Nature Publishing Group},
title = {{Circuit quantum electrodynamics with a spin qubit}},
url = {http://dx.doi.org/10.1038/nature11559},
volume = {490},
year = {2012}
}
@article{Tosi2014,
abstract = {Recent advances in silicon nanofabrication have allowed the manipulation of spin qubits that are extremely isolated from noise sources, being therefore the semiconductor equivalent of single atoms in vacuum. We investigate the possibility of directly coupling an electron spin qubit to a superconducting resonator magnetic vacuum field. By using resonators modified to increase the vacuum magnetic field at the qubit location, and isotopically purified 28Si substrates, it is possible to achieve coupling rates faster than the single spin dephasing. This opens up new avenues for circuit-quantum electrodynamics with spins, and provides a pathway for dispersive read-out of spin qubits via superconducting resonators.},
archivePrefix = {arXiv},
arxivId = {arXiv:1405.1231v1},
author = {Tosi, Guilherme and Mohiyaddin, Fahd a. and Huebl, Hans and Morello, Andrea},
doi = {10.1063/1.4893242},
eprint = {arXiv:1405.1231v1},
file = {:C$\backslash$:/Users/Steffi/AppData/Local/Mendeley Ltd./Mendeley Desktop/Downloaded/Tosi et al. - 2014 - Circuit-quantum electrodynamics with direct magnetic coupling to single-atom spin qubits in isotopically enriched 2.pdf:pdf},
issn = {2158-3226},
journal = {AIP Advances},
number = {8},
pages = {087122},
title = {{Circuit-quantum electrodynamics with direct magnetic coupling to single-atom spin qubits in isotopically enriched 28Si}},
url = {http://scitation.aip.org/content/aip/journal/adva/4/8/10.1063/1.4893242},
volume = {4},
year = {2014}
}
@article{Baugh2007,
author = {Baugh, Jonathan and Kitamura, Yosuke and Ono, Keiji and Tarucha, Seigo},
file = {:C$\backslash$:/Users/Steffi/AppData/Local/Mendeley Ltd./Mendeley Desktop/Downloaded/Baugh et al. - 2007 - Large Nuclear Overhauser Fields Detected in Vertically Coupled Double Quantum Dots.pdf:pdf},
journal = {Physical Review Letters},
pages = {096804},
title = {{Large Nuclear Overhauser Fields Detected in Vertically Coupled Double Quantum Dots}},
url = {http://journals.aps.org/prl/pdf/10.1103/PhysRevLett.99.096804},
volume = {99},
year = {2007}
}
@article{Petta2008,
author = {Petta, J. R. and Taylor, J. M. and Johnson, A. C. and Yacoby, A. and Lukin, M. D. and Marcus, C. M. and Hanson, M. P. and Gossard, A. C.},
journal = {Physical Review Letters},
pages = {067601},
title = {{Dynamic Nuclear Polarization with Single Electron Spins}},
url = {http://pettagroup.princeton.edu/publications/2008/PRL{\_}100{\_}067601{\_}2008.pdf},
volume = {100},
year = {2008}
}
@article{Chen2011,
abstract = {We report a technique for applying a dc voltage or current bias to the center conductor of a high-quality factor superconductingmicrowavecavity without significantly disturbing selected cavity modes. This is accomplished by incorporating dc bias lines into the cavity at specific locations. The measured S -matrix parameters of the system are in good agreement with theoretical predictions and simulations. We find that at 4 K the quality factor of the cavity degrades by less than 1{\%} under the application of a dc bias.},
author = {Chen, Fei and Sirois, A. J. and Simmonds, R. W. and Rimberg, A. J.},
doi = {10.1063/1.3573824},
issn = {00036951},
journal = {Applied Physics Letters},
month = {mar},
number = {13},
pages = {132509},
publisher = {AIP Publishing},
title = {{Introduction of a dc bias into a high-Q superconducting microwave cavity}},
url = {http://scitation.aip.org/content/aip/journal/apl/98/13/10.1063/1.3573824},
volume = {98},
year = {2011}
}
@article{Tartakovskii2007,
abstract = {We show that by illuminating an InGaAs/GaAs self-assembled quantum dot with circularly polarized light, the nuclei of atoms constituting the dot can be driven into a bistable regime, in which either a thresholdlike enhancement or reduction of the local nuclear field by up to 3 T can be generated by varying the pumping intensity. The excitation power threshold for such a nuclear spin "switch" is found to depend on both the external magnetic and electric fields. The switch is shown to arise from the strong feedback of the nuclear spin polarization on the dynamics of the spin transfer from electrons to the nuclei of the dot.},
author = {Tartakovskii, a. I. and Wright, T. and Russell, a. and Fal'Ko, V. I. and Van'Kov, a. B. and Skiba-Szymanska, J. and Drouzas, I. and Kolodka, R. S. and Skolnick, M. S. and Fry, P. W. and Tahraoui, a. and Liu, H. Y. and Hopkinson, M.},
doi = {10.1103/PhysRevLett.98.026806},
file = {:C$\backslash$:/Users/Steffi/AppData/Local/Mendeley Ltd./Mendeley Desktop/Downloaded/Tartakovskii et al. - 2007 - Nuclear spin switch in semiconductor quantum dots.pdf:pdf},
isbn = {0031-9007},
issn = {00319007},
journal = {Physical Review Letters},
number = {JANUARY},
pages = {1--4},
pmid = {17358634},
title = {{Nuclear spin switch in semiconductor quantum dots}},
volume = {98},
year = {2007}
}
@article{Braun2005,
author = {Braun, P. F. and Marie, X. and Lombez, L. and Urbaszek, B. and Amand, T. and Renucci, P. and Kalevich, V. K. and Kavokin, K. V. and Krebs, O. and Voisin, P. and Masumoto, Y.},
doi = {10.1103/PhysRevLett.94.116601},
file = {:C$\backslash$:/Users/Steffi/AppData/Local/Mendeley Ltd./Mendeley Desktop/Downloaded/Braun et al. - 2005 - Direct observation of the electron spin relaxation induced by nuclei in quantum dots.pdf:pdf},
issn = {00319007},
journal = {Physical Review Letters},
number = {March},
pages = {1--4},
title = {{Direct observation of the electron spin relaxation induced by nuclei in quantum dots}},
volume = {94},
year = {2005}
}
@article{Ono2004,
abstract = {We show experimentally that electron transport through GaAs-based double quantum dots can be affected by ambient nuclear spin states in a certain regime where transport is blocked in the absence of electron spin flip. Current through the dots oscillates in time with a period up to 200 s depending on magnetic field. Oscillation is quenched by application of a continuous wave ac magnetic field which can induce nuclear magnetic resonance in 71Ga or 69Ga. A possible mechanism for dynamically polarizing the nuclear spins is proposed.},
author = {Ono, Keiji and Tarucha, Seigo},
doi = {10.1103/PhysRevLett.92.256803},
file = {:C$\backslash$:/Users/Steffi/AppData/Local/Mendeley Ltd./Mendeley Desktop/Downloaded/Ono, Tarucha - 2004 - Nuclear-spin-induced oscillatory current in spin-blockaded quantum dots.pdf:pdf},
isbn = {0031-9007},
issn = {00319007},
journal = {Physical Review Letters},
number = {June},
pages = {256803--1},
pmid = {15245046},
title = {{Nuclear-spin-induced oscillatory current in spin-blockaded quantum dots}},
volume = {92},
year = {2004}
}
@article{Yu2013,
author = {Yu, Hong-Yi and Luo, Yu and Yao, Wang},
doi = {10.1088/0256-307X/30/7/077302},
file = {:C$\backslash$:/Users/Steffi/AppData/Local/Mendeley Ltd./Mendeley Desktop/Downloaded/Yu, Luo, Yao - 2013 - The Nuclear Dark State under Dynamical Nuclear Polarization.pdf:pdf},
issn = {0256-307X},
journal = {Chinese Physics Letters},
pages = {077302},
title = {{The Nuclear Dark State under Dynamical Nuclear Polarization}},
url = {http://stacks.iop.org/0256-307X/30/i=7/a=077302?key=crossref.b668e66cc27d49cb0c9d5e87666a7f4d},
volume = {30},
year = {2013}
}
@article{Wald1994,
author = {Wald, Kr and Kouwenhoven, Lp},
doi = {10.1103/PhysRevLett.73.1011},
file = {:C$\backslash$:/Users/Steffi/AppData/Local/Mendeley Ltd./Mendeley Desktop/Downloaded/Wald, Kouwenhoven - 1994 - Local dynamic nuclear polarization using quantum point contacts.pdf:pdf},
isbn = {00319007},
issn = {0031-9007},
journal = {Physical review {\ldots}},
number = {7},
pages = {1011--1015},
title = {{Local dynamic nuclear polarization using quantum point contacts}},
url = {http://journals.aps.org/prl/abstract/10.1103/PhysRevLett.73.1011},
volume = {73},
year = {1994}
}
@article{Maletinsky2007,
abstract = {We present measurements of the buildup and decay of nuclear spin polarization in a single semiconductor quantum dot. Our experiment shows that we polarize the nuclei in a few milliseconds, while their decay dynamics depends drastically on external parameters. We show that a single electron can very efficiently depolarize nuclear spins in milliseconds whereas in the absence of the electron the nuclear spin lifetime is on the scale of seconds. This lifetime is further enhanced by 1-2 orders of magnitude by quenching the nonsecular nuclear dipole-dipole interactions with a magnetic field of 1 mT.},
author = {Maletinsky, P. and Badolato, a. and Imamoglu, a.},
doi = {10.1103/PhysRevLett.99.056804},
file = {:C$\backslash$:/Users/Steffi/AppData/Local/Mendeley Ltd./Mendeley Desktop/Downloaded/Maletinsky, Badolato, Imamoglu - 2007 - Dynamics of quantum dot nuclear spin polarization controlled by a single electron.pdf:pdf},
issn = {00319007},
journal = {Physical Review Letters},
number = {August},
pages = {1--4},
pmid = {17930778},
title = {{Dynamics of quantum dot nuclear spin polarization controlled by a single electron}},
volume = {99},
year = {2007}
}
@article{Bracker2005,
abstract = {We present a comprehensive examination of optical pumping of spins in individual GaAs quantum dots as we change the net charge from positive to neutral to negative with a charge-tunable heterostructure. Negative photoluminescence polarization memory is enhanced by optical pumping of ground state electron spins, which we prove with the first measurements of the Hanle effect on an individual quantum dot. We use the Overhauser effect in a high longitudinal magnetic field to demonstrate efficient optical pumping of nuclear spins for all three charge states of the quantum dot.},
archivePrefix = {arXiv},
arxivId = {cond-mat/0408466},
author = {Bracker, a. S. and Stinaff, E. a. and Gammon, D. and Ware, M. E. and Tischler, J. G. and Shabaev, a. and Efros, Al L. and Park, D. and Gershoni, D. and Korenev, V. L. and Merkulov, I. a.},
doi = {10.1103/PhysRevLett.94.047402},
eprint = {0408466},
file = {:C$\backslash$:/Users/Steffi/AppData/Local/Mendeley Ltd./Mendeley Desktop/Downloaded/Bracker et al. - 2005 - Optical pumping of the electronic and nuclear spin of single charge-tunable quantum dots.pdf:pdf},
isbn = {0031-9007},
issn = {00319007},
journal = {Physical Review Letters},
number = {February},
pages = {1--4},
pmid = {15783594},
primaryClass = {cond-mat},
title = {{Optical pumping of the electronic and nuclear spin of single charge-tunable quantum dots}},
volume = {94},
year = {2005}
}
@article{Koppens2006,
abstract = {The ability to control the quantum state of a single electron spin in a quantum dot is at the heart of recent developments towards a scalable spin-based quantum computer. In combination with the recently demonstrated controlled exchange gate between two neighbouring spins, driven coherent single spin rotations would permit universal quantum operations. Here, we report the experimental realization of single electron spin rotations in a double quantum dot. First, we apply a continuous-wave oscillating magnetic field, generated on-chip, and observe electron spin resonance in spin-dependent transport measurements through the two dots. Next, we coherently control the quantum state of the electron spin by applying short bursts of the oscillating magnetic field and observe about eight oscillations of the spin state (so-called Rabi oscillations) during a microsecond burst. These results demonstrate the feasibility of operating single-electron spins in a quantum dot as quantum bits.},
author = {Koppens, F H L and Buizert, C and Tielrooij, K J and Vink, I T and Nowack, K C and Meunier, T and Kouwenhoven, L P and Vandersypen, L M K},
doi = {10.1038/nature05065},
file = {:C$\backslash$:/Users/Steffi/AppData/Local/Mendeley Ltd./Mendeley Desktop/Downloaded/Koppens et al. - 2006 - Driven coherent oscillations of a single electron spin in a quantum dot.pdf:pdf},
isbn = {0028-0836},
issn = {0028-0836},
journal = {Nature},
number = {August},
pages = {766--771},
pmid = {16915280},
title = {{Driven coherent oscillations of a single electron spin in a quantum dot.}},
volume = {442},
year = {2006}
}
@article{Pang2015,
author = {Pang, Hongliang and Gong, Zhirui and Yao, Wang},
doi = {10.1103/PhysRevB.91.035305},
file = {:C$\backslash$:/Users/Steffi/AppData/Local/Mendeley Ltd./Mendeley Desktop/Downloaded/Pang, Gong, Yao - 2015 - Feedback control of nuclear spin bath of a single hole spin in a quantum dot.pdf:pdf},
pages = {1--10},
title = {{Feedback control of nuclear spin bath of a single hole spin in a quantum dot}},
volume = {035305},
year = {2015}
}
@article{Reilly2011,
author = {Reilly, D J},
doi = {10.1126/science.1159221},
file = {:C$\backslash$:/Users/Steffi/AppData/Local/Mendeley Ltd./Mendeley Desktop/Downloaded/Reilly - 2011 - Suppressing Spin Qubit Dephasing by.pdf:pdf},
journal = {Icarus},
number = {2000},
title = {{Suppressing Spin Qubit Dephasing by}},
volume = {817},
year = {2011}
}
@misc{,
title = {{Theory of one and two donors in Si}},
url = {http://arxiv.org/pdf/1407.8224v1.pdf},
urldate = {2015-03-06}
}
@article{Neder2011,
abstract = {We study electron spin decoherence in a two-electron double quantum dot due to the hyperfine interaction, under spin-echo conditions as studied in recent experiments. We develop a semi-classical model for the interaction between the electron and nuclear spins, in which the time-dependent Overhauser fields induced by the nuclear spins are treated as classical vector variables. Comparison of the model with experimentally-obtained echo signals allows us to quantify the contributions of various processes such as coherent Larmor precession and spin diffusion to the nuclear spin evolution.},
archivePrefix = {arXiv},
arxivId = {1103.4862},
author = {Neder, Izhar and Rudner, Mark S. and Bluhm, Hendrik and Foletti, Sandra and Halperin, Bertrand I. and Yacoby, Amir},
doi = {10.1103/PhysRevB.84.035441},
eprint = {1103.4862},
file = {:C$\backslash$:/Users/Steffi/AppData/Local/Mendeley Ltd./Mendeley Desktop/Downloaded/Neder et al. - 2011 - Semiclassical model for the dephasing of a two-electron spin qubit coupled to a coherently evolving nuclear spin b.pdf:pdf},
issn = {10980121},
journal = {Physical Review B - Condensed Matter and Materials Physics},
title = {{Semiclassical model for the dephasing of a two-electron spin qubit coupled to a coherently evolving nuclear spin bath}},
volume = {84},
year = {2011}
}
@misc{,
keywords = {T. Duty, G. Johansson, K. Bladh, D. Gunnarsson, C.},
title = {{Observation of Quantum Capacitance in the Cooper-Pair Transistor}},
url = {http://journals.aps.org/prl/pdf/10.1103/PhysRevLett.95.206807},
urldate = {2015-02-24}
}
@article{Morello2010,
abstract = {The size of silicon transistors used in microelectronic devices is shrinking to the level at which quantum effects become important. Although this presents a significant challenge for the further scaling of microprocessors, it provides the potential for radical innovations in the form of spin-based quantum computers and spintronic devices. An electron spin in silicon can represent a well-isolated quantum bit with long coherence times because of the weak spin-orbit coupling and the possibility of eliminating nuclear spins from the bulk crystal. However, the control of single electrons in silicon has proved challenging, and so far the observation and manipulation of a single spin has been impossible. Here we report the demonstration of single-shot, time-resolved readout of an electron spin in silicon. This has been performed in a device consisting of implanted phosphorus donors coupled to a metal-oxide-semiconductor single-electron transistor-compatible with current microelectronic technology. We observed a spin lifetime of ∼6 seconds at a magnetic field of 1.5 tesla, and achieved a spin readout fidelity better than 90 per cent. High-fidelity single-shot spin readout in silicon opens the way to the development of a new generation of quantum computing and spintronic devices, built using the most important material in the semiconductor industry.},
author = {Morello, Andrea and Pla, Jarryd J and Zwanenburg, Floris A and Chan, Kok W and Tan, Kuan Y and Huebl, Hans and M{\"{o}}tt{\"{o}}nen, Mikko and Nugroho, Christopher D and Yang, Changyi and van Donkelaar, Jessica A and Alves, Andrew D C and Jamieson, David N and Escott, Christopher C and Hollenberg, Lloyd C L and Clark, Robert G and Dzurak, Andrew S},
doi = {10.1038/nature09392},
file = {::},
issn = {1476-4687},
journal = {Nature},
month = {oct},
number = {7316},
pages = {687--91},
pmid = {20877281},
publisher = {Nature Publishing Group, a division of Macmillan Publishers Limited. All Rights Reserved.},
shorttitle = {Nature},
title = {{Single-shot readout of an electron spin in silicon.}},
url = {http://dx.doi.org/10.1038/nature09392},
volume = {467},
year = {2010}
}
@misc{TheMendeleySupportTeam2011a,
abstract = {A quick introduction to Mendeley. Learn how Mendeley creates your personal digital library, how to organize and annotate documents, how to collaborate and share with colleagues, and how to generate citations and bibliographies.},
address = {London},
author = {{The Mendeley Support Team}},
booktitle = {Mendeley Desktop},
keywords = {Mendeley,how-to,user manual},
pages = {1--16},
publisher = {Mendeley Ltd.},
title = {{Getting Started with Mendeley}},
url = {http://www.mendeley.com},
year = {2011}
}
@article{Shulman2014a,
abstract = {Unwanted interaction between a quantum system and its fluctuating environment leads to decoherence and is the primary obstacle to establishing a scalable quantum information processing architecture. Strategies such as environmental and materials engineering, quantum error correction and dynamical decoupling can mitigate decoherence, but generally increase experimental complexity. Here we improve coherence in a qubit using real-time Hamiltonian parameter estimation. Using a rapidly converging Bayesian approach, we precisely measure the splitting in a singlet-triplet spin qubit faster than the surrounding nuclear bath fluctuates. We continuously adjust qubit control parameters based on this information, thereby improving the inhomogenously broadened coherence time (T2*) from tens of nanoseconds to {\textgreater}2 $\mu$s. Because the technique demonstrated here is compatible with arbitrary qubit operations, it is a natural complement to quantum error correction and can be used to improve the performance of a wide variety of qubits in both meteorological and quantum information processing applications.},
author = {Shulman, M D and Harvey, S P and Nichol, J M and Bartlett, S D and Doherty, a C and Umansky, V and Yacoby, a},
doi = {10.1038/ncomms6156},
file = {:C$\backslash$:/Users/Steffi/AppData/Local/Mendeley Ltd./Mendeley Desktop/Downloaded/Shulman et al. - 2014 - Suppressing qubit dephasing using real-time Hamiltonian estimation.pdf:pdf},
issn = {2041-1723},
journal = {Nature communications},
month = {jan},
number = {May},
pages = {5156},
pmid = {25295674},
publisher = {Nature Publishing Group},
title = {{Suppressing qubit dephasing using real-time Hamiltonian estimation.}},
url = {http://www.ncbi.nlm.nih.gov/pubmed/25295674},
volume = {5},
year = {2014}
}
@article{Barnes2012,
author = {Barnes, Edwin and Cywi{\'{n}}ski, {\L}ukasz and {Das Sarma}, S.},
doi = {10.1103/PhysRevLett.109.140403},
file = {:C$\backslash$:/Users/Steffi/AppData/Local/Mendeley Ltd./Mendeley Desktop/Downloaded/Barnes, Cywi{\'{n}}ski, Das Sarma - 2012 - Nonperturbative Master Equation Solution of Central Spin Dephasing Dynamics.pdf:pdf},
issn = {0031-9007},
journal = {Physical Review Letters},
month = {oct},
number = {14},
pages = {140403},
title = {{Nonperturbative Master Equation Solution of Central Spin Dephasing Dynamics}},
url = {http://link.aps.org/doi/10.1103/PhysRevLett.109.140403},
volume = {109},
year = {2012}
}
@article{Rudner2007,
author = {Rudner, M. and Levitov, L.},
doi = {10.1103/PhysRevLett.99.246602},
file = {:C$\backslash$:/Users/Steffi/AppData/Local/Mendeley Ltd./Mendeley Desktop/Downloaded/Rudner, Levitov - 2007 - Electrically Driven Reverse Overhauser Pumping of Nuclear Spins in Quantum Dots.pdf:pdf},
issn = {0031-9007},
journal = {Physical Review Letters},
month = {dec},
number = {24},
pages = {246602},
title = {{Electrically Driven Reverse Overhauser Pumping of Nuclear Spins in Quantum Dots}},
url = {http://link.aps.org/doi/10.1103/PhysRevLett.99.246602},
volume = {99},
year = {2007}
}
@article{Danon2009a,
author = {Danon, J. and Vink, I. and Koppens, F. and Nowack, K. and Vandersypen, L. and Nazarov, Yu.},
doi = {10.1103/PhysRevLett.103.046601},
file = {:C$\backslash$:/Users/Steffi/AppData/Local/Mendeley Ltd./Mendeley Desktop/Downloaded/Danon et al. - 2009 - Multiple Nuclear Polarization States in a Double Quantum Dot.pdf:pdf},
issn = {0031-9007},
journal = {Physical Review Letters},
month = {jul},
number = {4},
pages = {046601},
title = {{Multiple Nuclear Polarization States in a Double Quantum Dot}},
url = {http://link.aps.org/doi/10.1103/PhysRevLett.103.046601},
volume = {103},
year = {2009}
}
@article{Economou2014,
author = {Economou, Sophia E and Barnes, Edwin},
doi = {10.1103/PhysRevB.89.165301},
file = {:C$\backslash$:/Users/Steffi/AppData/Local/Mendeley Ltd./Mendeley Desktop/Downloaded/Economou, Barnes - 2014 - Theory of dynamic nuclear polarization and feedback in quantum dots.pdf:pdf},
pages = {1--22},
title = {{Theory of dynamic nuclear polarization and feedback in quantum dots}},
volume = {165301},
year = {2014}
}
@article{Yang2013,
author = {Yang, Wen and Sham, L J},
doi = {10.1103/PhysRevB.88.235304},
file = {:C$\backslash$:/Users/Steffi/AppData/Local/Mendeley Ltd./Mendeley Desktop/Downloaded/Yang, Sham - 2013 - General theory of feedback control of a nuclear spin ensemble in quantum dots.pdf:pdf},
pages = {75--78},
title = {{General theory of feedback control of a nuclear spin ensemble in quantum dots}},
volume = {235304},
year = {2013}
}
@article{Search,
author = {{Gong, Zhe-Xuan Yian, Zhang-qi Duan}, L-M},
doi = {10.1088/1367-2630/13/3/033036},
file = {:C$\backslash$:/Users/Steffi/AppData/Local/Mendeley Ltd./Mendeley Desktop/Downloaded/Yang, Sham - 2013 - General theory of feedback control of a nuclear spin ensemble in quantum dots.pdf:pdf},
title = {{Dynamics of the Overhauser field under nuclear spin diffusion in a quantum dot}},
volume = {033036}
}
@article{Danon2008,
author = {Danon, Jeroen and Nazarov, Yuli V},
doi = {10.1103/PhysRevLett.100.056603},
file = {:C$\backslash$:/Users/Steffi/AppData/Local/Mendeley Ltd./Mendeley Desktop/Downloaded/Danon, Nazarov - 2008 - Nuclear Tuning and Detuning of the Electron Spin Resonance in a Quantum Dot.pdf:pdf},
number = {February},
pages = {1--4},
title = {{Nuclear Tuning and Detuning of the Electron Spin Resonance in a Quantum Dot :}},
volume = {056603},
year = {2008}
}
@article{Yang2012,
author = {Yang, Wen and Sham, L J},
doi = {10.1103/PhysRevB.85.235319},
file = {:C$\backslash$:/Users/Steffi/AppData/Local/Mendeley Ltd./Mendeley Desktop/Downloaded/Yang, Sham - 2012 - Collective nuclear stabilization in single quantum dots by noncollinear hyperfine interaction.pdf:pdf},
number = {December 2010},
pages = {1--7},
title = {{Collective nuclear stabilization in single quantum dots by noncollinear hyperfine interaction}},
volume = {235319},
year = {2012}
}
@article{Coish2010,
author = {Coish, W. a. and Fischer, Jan and Loss, Daniel},
doi = {10.1103/PhysRevB.81.165315},
file = {:C$\backslash$:/Users/Steffi/AppData/Local/Mendeley Ltd./Mendeley Desktop/Downloaded/Coish, Fischer, Loss - 2010 - Free-induction decay and envelope modulations in a narrowed nuclear spin bath.pdf:pdf},
issn = {1098-0121},
journal = {Physical Review B},
month = {apr},
number = {16},
pages = {165315},
title = {{Free-induction decay and envelope modulations in a narrowed nuclear spin bath}},
url = {http://link.aps.org/doi/10.1103/PhysRevB.81.165315},
volume = {81},
year = {2010}
}
@article{Witzel2014,
author = {Witzel, Wayne M. and Young, Kevin and {Das Sarma}, Sankar},
doi = {10.1103/PhysRevB.90.115431},
file = {:C$\backslash$:/Users/Steffi/AppData/Local/Mendeley Ltd./Mendeley Desktop/Downloaded/Witzel, Young, Das Sarma - 2014 - Converting a real quantum spin bath to an effective classical noise acting on a central spin.pdf:pdf},
issn = {1098-0121},
journal = {Physical Review B},
month = {sep},
number = {11},
pages = {115431},
title = {{Converting a real quantum spin bath to an effective classical noise acting on a central spin}},
url = {http://link.aps.org/doi/10.1103/PhysRevB.90.115431},
volume = {90},
year = {2014}
}
@article{Schreiber2011,
abstract = {Artificial molecules containing just one or two electrons provide a powerful platform for studies of orbital and spin quantum dynamics in nanoscale devices. A well-known example of these dynamics is tunnelling of electrons between two coupled quantum dots triggered by microwave irradiation. So far, these tunnelling processes have been treated as electric-dipole-allowed spin-conserving events. Here we report that microwaves can also excite tunnelling transitions between states with different spin. We show that the dominant mechanism responsible for violation of spin conservation is the spin-orbit interaction. These transitions make it possible to perform detailed microwave spectroscopy of the molecular spin states of an artificial hydrogen molecule and open up the possibility of realizing full quantum control of a two-spin system through microwave excitation.},
author = {Schreiber, L R and Braakman, F R and Meunier, T and Calado, V and Danon, J and Taylor, J M and Wegscheider, W and Vandersypen, L M K},
doi = {10.1038/ncomms1561},
file = {:C$\backslash$:/Users/Steffi/AppData/Local/Mendeley Ltd./Mendeley Desktop/Downloaded/Schreiber et al. - 2011 - Coupling artificial molecular spin states by photon-assisted tunnelling.pdf:pdf},
issn = {2041-1723},
journal = {Nature communications},
month = {jan},
pages = {556},
pmid = {22109530},
publisher = {Nature Publishing Group},
title = {{Coupling artificial molecular spin states by photon-assisted tunnelling.}},
url = {http://www.pubmedcentral.nih.gov/articlerender.fcgi?artid=3483534{\&}tool=pmcentrez{\&}rendertype=abstract},
volume = {2},
year = {2011}
}
@misc{sm2014,
title = {special-measure},
url = {https://code.google.com/p/special-measure/}
}
@book{Bronstein2008,
author = {Bronstein, I.N.},
title = {{Taschenbuch der Mathematik}},
year = {2008}
}
@article{Witzel2008,
author = {Witzel, W. M. and {Das Sarma}, S.},
doi = {10.1103/PhysRevB.77.165319},
file = {:C$\backslash$:/Users/Steffi/AppData/Local/Mendeley Ltd./Mendeley Desktop/Downloaded/Witzel, Das Sarma - 2008 - Wavefunction considerations for the central spin decoherence problem in a nuclear spin bath.pdf:pdf},
issn = {1098-0121},
journal = {Physical Review B},
month = {apr},
number = {16},
pages = {165319},
title = {{Wavefunction considerations for the central spin decoherence problem in a nuclear spin bath}},
url = {http://link.aps.org/doi/10.1103/PhysRevB.77.165319},
volume = {77},
year = {2008}
}
@article{Shulman2012a,
abstract = {Quantum computers have the potential to solve certain problems faster than classical computers. To exploit their power, it is necessary to perform interqubit operations and generate entangled states. Spin qubits are a promising candidate for implementing a quantum processor because of their potential for scalability and miniaturization. However, their weak interactions with the environment, which lead to their long coherence times, make interqubit operations challenging. We performed a controlled two-qubit operation between singlet-triplet qubits using a dynamically decoupled sequence that maintains the two-qubit coupling while decoupling each qubit from its fluctuating environment. Using state tomography, we measured the full density matrix of the system and determined the concurrence and the fidelity of the generated state, providing proof of entanglement.},
author = {Shulman, M D and Dial, O E and Harvey, S P and Bluhm, H and Umansky, V and Yacoby, A},
doi = {10.1126/science.1217692},
file = {:C$\backslash$:/Users/Steffi/AppData/Local/Mendeley Ltd./Mendeley Desktop/Downloaded/Shulman et al. - 2012 - Demonstration of entanglement of electrostatically coupled singlet-triplet qubits.pdf:pdf},
issn = {1095-9203},
journal = {Science},
month = {apr},
number = {6078},
pages = {202--5},
pmid = {22499942},
title = {{Demonstration of entanglement of electrostatically coupled singlet-triplet qubits.}},
url = {http://www.ncbi.nlm.nih.gov/pubmed/22499942},
volume = {336},
year = {2012}
}
@article{,
title = {{Special Measure}},
url = {http://code.google.com/p/special-measure/}
}
@article{Rashba2008a,
author = {Rashba, Emmanuel},
doi = {10.1103/PhysRevB.78.195302},
file = {:C$\backslash$:/Users/Steffi/AppData/Local/Mendeley Ltd./Mendeley Desktop/Downloaded/Rashba - 2008 - Theory of electric dipole spin resonance in quantum dots Mean field theory with Gaussian fluctuations and beyond.pdf:pdf},
issn = {1098-0121},
journal = {Physical Review B},
month = {nov},
number = {19},
pages = {195302},
title = {{Theory of electric dipole spin resonance in quantum dots: Mean field theory with Gaussian fluctuations and beyond}},
url = {http://link.aps.org/doi/10.1103/PhysRevB.78.195302},
volume = {78},
year = {2008}
}
@article{Yao2006,
author = {Yao, Wang and Liu, Ren-Bao and Sham, L.},
doi = {10.1103/PhysRevB.74.195301},
file = {:C$\backslash$:/Users/Steffi/AppData/Local/Mendeley Ltd./Mendeley Desktop/Downloaded/Yao, Liu, Sham - 2006 - Theory of electron spin decoherence by interacting nuclear spins in a quantum dot.pdf:pdf},
issn = {1098-0121},
journal = {Physical Review B},
month = {nov},
number = {19},
pages = {195301},
title = {{Theory of electron spin decoherence by interacting nuclear spins in a quantum dot}},
url = {http://link.aps.org/doi/10.1103/PhysRevB.74.195301},
volume = {74},
year = {2006}
}
@article{Schreiner1997,
author = {Schreiner, M.},
file = {:C$\backslash$:/Users/Steffi/AppData/Local/Mendeley Ltd./Mendeley Desktop/Downloaded/Schreiner - 1997 - 1997 by h4.pdf:pdf},
number = {10},
pages = {715--720},
title = {1997 by h4.},
volume = {102},
year = {1997}
}
@article{Shulman2014,
archivePrefix = {arXiv},
arxivId = {1405.0485},
author = {Shulman, Michael D. and Harvey, Shannon P. and Nichol, John M. and Bartlett, Stephen D. and Doherty, Andrew C. and Umansky, Vladimir and Yacoby, Amir},
eprint = {1405.0485},
file = {:C$\backslash$:/Users/Steffi/AppData/Local/Mendeley Ltd./Mendeley Desktop/Downloaded/Shulman et al. - 2014 - Suppressing qubit dephasing using real-time Hamiltonian estimation(2).pdf:pdf},
month = {may},
pages = {1--8},
title = {{Suppressing qubit dephasing using real-time Hamiltonian estimation}},
url = {http://arxiv.org/abs/1405.0485v1},
year = {2014}
}
@phdthesis{Humpohl2014,
author = {Humpohl, Simon Sebastian},
file = {:C$\backslash$:/Users/Steffi/AppData/Local/Mendeley Ltd./Mendeley Desktop/Downloaded/Humpohl - 2014 - A multithreaded data acquisition dll by.pdf:pdf},
number = {July},
title = {{A multithreaded data acquisition dll by}},
year = {2014}
}
@article{Hogele2012,
author = {H{\"{o}}gele, a. and Kroner, M. and Latta, C. and Claassen, M. and Carusotto, I. and Bulutay, C. and Imamoglu, a.},
doi = {10.1103/PhysRevLett.108.197403},
file = {:C$\backslash$:/Users/Steffi/AppData/Local/Mendeley Ltd./Mendeley Desktop/Downloaded/H{\"{o}}gele et al. - 2012 - Dynamic Nuclear Spin Polarization in the Resonant Laser Excitation of an InGaAs Quantum Dot.pdf:pdf},
issn = {0031-9007},
journal = {Physical Review Letters},
month = {may},
number = {19},
pages = {197403},
title = {{Dynamic Nuclear Spin Polarization in the Resonant Laser Excitation of an InGaAs Quantum Dot}},
url = {http://link.aps.org/doi/10.1103/PhysRevLett.108.197403},
volume = {108},
year = {2012}
}
@article{Gulde2003,
abstract = {Determining classically whether a coin is fair (head on one side, tail on the other) or fake (heads or tails on both sides) requires an examination of each side. However, the analogous quantum procedure (the Deutsch-Jozsa algorithm) requires just one examination step. The Deutsch-Jozsa algorithm has been realized experimentally using bulk nuclear magnetic resonance techniques, employing nuclear spins as quantum bits (qubits). In contrast, the ion trap processor utilises motional and electronic quantum states of individual atoms as qubits, and in principle is easier to scale to many qubits. Experimental advances in the latter area include the realization of a two-qubit quantum gate, the entanglement of four ions, quantum state engineering and entanglement-enhanced phase estimation. Here we exploit techniques developed for nuclear magnetic resonance to implement the Deutsch-Jozsa algorithm on an ion-trap quantum processor, using as qubits the electronic and motional states of a single calcium ion. Our ion-based implementation of a full quantum algorithm serves to demonstrate experimental procedures with the quality and precision required for complex computations, confirming the potential of trapped ions for quantum computation.},
author = {Gulde, Stephan and Riebe, Mark and Lancaster, Gavin P T and Becher, Christoph and Eschner, J{\"{u}}rgen and H{\"{a}}ffner, Hartmut and Schmidt-Kaler, Ferdinand and Chuang, Isaac L and Blatt, Rainer},
doi = {10.1038/nature01336},
file = {:C$\backslash$:/Users/Steffi/AppData/Local/Mendeley Ltd./Mendeley Desktop/Downloaded/Gulde et al. - 2003 - Implementation of the Deutsch-Jozsa algorithm on an ion-trap quantum computer.pdf:pdf},
issn = {0028-0836},
journal = {Nature},
month = {jan},
number = {6918},
pages = {48--50},
pmid = {12511949},
title = {{Implementation of the Deutsch-Jozsa algorithm on an ion-trap quantum computer.}},
url = {http://www.ncbi.nlm.nih.gov/pubmed/12511949},
volume = {421},
year = {2003}
}
@article{Buluta2009,
abstract = {Quantum simulators are controllable quantum systems that can be used to simulate other quantum systems. Being able to tackle problems that are intractable on classical computers, quantum simulators would provide a means of exploring new physical phenomena. We present an overview of how quantum simulators may become a reality in the near future as the required technologies are now within reach. Quantum simulators, relying on the coherent control of neutral atoms, ions, photons, or electrons, would allow studying problems in various fields including condensed-matter physics, high-energy physics, cosmology, atomic physics, and quantum chemistry.},
author = {Buluta, Iulia and Nori, Franco},
doi = {10.1126/science.1177838},
file = {:C$\backslash$:/Users/Steffi/AppData/Local/Mendeley Ltd./Mendeley Desktop/Downloaded/Buluta, Nori - 2009 - Quantum simulators.pdf:pdf},
issn = {1095-9203},
journal = {Science},
month = {oct},
number = {5949},
pages = {108--11},
pmid = {19797653},
title = {{Quantum simulators.}},
url = {http://www.ncbi.nlm.nih.gov/pubmed/19797653},
volume = {326},
year = {2009}
}
@article{Lo2014,
author = {Lo, Hoi-Kwong and Curty, Marcos and Tamaki, Kiyoshi},
doi = {10.1038/nphoton.2014.149},
file = {:C$\backslash$:/Users/Steffi/AppData/Local/Mendeley Ltd./Mendeley Desktop/Downloaded/Lo, Curty, Tamaki - 2014 - Secure quantum key distribution.pdf:pdf},
issn = {1749-4885},
journal = {Nature Photonics},
month = {jul},
number = {8},
pages = {595--604},
publisher = {Nature Publishing Group},
title = {{Secure quantum key distribution}},
url = {http://www.nature.com/doifinder/10.1038/nphoton.2014.149},
volume = {8},
year = {2014}
}
@article{Xu2009,
abstract = {A single electron or hole spin trapped inside a semiconductor quantum dot forms the foundation for many proposed quantum logic devices. In group III-V materials, the resonance and coherence between two ground states of the single spin are inevitably affected by the lattice nuclear spins through the hyperfine interaction, while the dynamics of the single spin also influence the nuclear environment. Recent efforts have been made to protect the coherence of spins in quantum dots by suppressing the nuclear spin fluctuations. However, coherent control of a single spin in a single dot with simultaneous suppression of the nuclear fluctuations has yet to be achieved. Here we report the suppression of nuclear field fluctuations in a singly charged quantum dot to well below the thermal value, as shown by an enhancement of the single electron spin dephasing time T(2)*, which we measure using coherent dark-state spectroscopy. The suppression of nuclear fluctuations is found to result from a hole-spin assisted dynamic nuclear spin polarization feedback process, where the stable value of the nuclear field is determined only by the laser frequencies at fixed laser powers. This nuclear field locking is further demonstrated in a three-laser measurement, indicating a possible enhancement of the electron spin T(2)* by a factor of several hundred. This is a simple and powerful method of enhancing the electron spin coherence time without use of 'spin echo'-type techniques. We expect that our results will enable the reproducible preparation of the nuclear spin environment for repetitive control and measurement of a single spin with minimal statistical broadening.},
author = {Xu, Xiaodong and Yao, Wang and Sun, Bo and Steel, Duncan G and Bracker, Allan S and Gammon, Daniel and Sham, L J},
doi = {10.1038/nature08120},
file = {:C$\backslash$:/Users/Steffi/AppData/Local/Mendeley Ltd./Mendeley Desktop/Downloaded/Xu et al. - 2009 - Optically controlled locking of the nuclear field via coherent dark-state spectroscopy.pdf:pdf},
issn = {1476-4687},
journal = {Nature},
month = {jun},
number = {7250},
pages = {1105--9},
pmid = {19553994},
publisher = {Nature Publishing Group},
title = {{Optically controlled locking of the nuclear field via coherent dark-state spectroscopy.}},
url = {http://www.ncbi.nlm.nih.gov/pubmed/19553994},
volume = {459},
year = {2009}
}
@article{Cywinski2008,
author = {Cywi{\'{n}}ski, Lukasz and Lutchyn, Roman and Nave, Cody and {Das Sarma}, S.},
doi = {10.1103/PhysRevB.77.174509},
file = {:C$\backslash$:/Users/Steffi/AppData/Local/Mendeley Ltd./Mendeley Desktop/Downloaded/Cywi{\'{n}}ski et al. - 2008 - How to enhance dephasing time in superconducting qubits.pdf:pdf},
issn = {1098-0121},
journal = {Physical Review B},
month = {may},
number = {17},
pages = {174509},
title = {{How to enhance dephasing time in superconducting qubits}},
url = {http://link.aps.org/doi/10.1103/PhysRevB.77.174509},
volume = {77},
year = {2008}
}
@article{Stueckelberg1932,
author = {Stueckelberg, E.},
journal = {Helvetica Physica Acta},
number = {370},
title = {{No Title}},
volume = {5},
year = {1932}
}
@article{Majorana1932,
author = {Majorana, Etorre},
file = {:C$\backslash$:/Users/Steffi/AppData/Local/Mendeley Ltd./Mendeley Desktop/Downloaded/Majorana - 1932 - Un atom,o orientato in un campo magnetico lentamente {\~{}}ariabile segue, come {\~{}} noto, adiabaticamente la direzione, suppo.pdf:pdf},
number = {2},
title = {{Un atom,o orientato in un campo magnetico lentamente {\~{}}ariabile segue, come {\~{}} noto, adiabaticamente la direzione, supposta variabile, dcl campo. A cib si deve il fatto recentemente po,sto in evidenza che sottoponendo un raggio molecolare prv{\~{}}'eniente da un}},
year = {1932}
}
@article{Zener1932,
author = {Zener, C.},
doi = {10.1098/rspa.1932.0165},
file = {:C$\backslash$:/Users/Steffi/AppData/Local/Mendeley Ltd./Mendeley Desktop/Downloaded/Zener - 1932 - Non-Adiabatic Crossing of Energy Levels.pdf:pdf},
issn = {1364-5021},
journal = {Proceedings of the Royal Society A: Mathematical, Physical and Engineering Sciences},
month = {sep},
number = {833},
pages = {696--702},
title = {{Non-Adiabatic Crossing of Energy Levels}},
url = {http://rspa.royalsocietypublishing.org/cgi/doi/10.1098/rspa.1932.0165},
volume = {137},
year = {1932}
}
@article{Landau1932a,
author = {Landau, L.},
journal = {Physics of the Soviet Union},
number = {88},
title = {{On the theory of transfer of energy at collisions 1}},
volume = {1},
year = {1932}
}
@article{Landau1932,
author = {Landau, L.},
journal = {Physics of the Soviet Union 2},
number = {46},
title = {{On the theory of transfer of energy at collisions 2}},
volume = {1},
year = {1932}
}
@article{Khaetskii2002,
author = {Khaetskii, Alexander and Loss, Daniel and Glazman, Leonid},
doi = {10.1103/PhysRevLett.88.186802},
file = {:C$\backslash$:/Users/Steffi/AppData/Local/Mendeley Ltd./Mendeley Desktop/Downloaded/Khaetskii, Loss, Glazman - 2002 - Electron Spin Decoherence in Quantum Dots due to Interaction with Nuclei.pdf:pdf},
issn = {0031-9007},
journal = {Physical Review Letters},
month = {apr},
number = {18},
pages = {186802},
title = {{Electron Spin Decoherence in Quantum Dots due to Interaction with Nuclei}},
url = {http://link.aps.org/doi/10.1103/PhysRevLett.88.186802},
volume = {88},
year = {2002}
}
@book{Abragam1961,
author = {Abragam, A.},
title = {{The principles of nuclear magnetism}},
year = {1961}
}
@article{Petta2008,
author = {Petta, J. and Taylor, J. and Johnson, a. and Yacoby, a. and Lukin, M. and Marcus, C. and Hanson, M. and Gossard, a.},
doi = {10.1103/PhysRevLett.100.067601},
file = {:C$\backslash$:/Users/Steffi/AppData/Local/Mendeley Ltd./Mendeley Desktop/Downloaded/Petta et al. - 2008 - Dynamic Nuclear Polarization with Single Electron Spins.pdf:pdf},
issn = {0031-9007},
journal = {Physical Review Letters},
month = {feb},
number = {6},
pages = {067601},
title = {{Dynamic Nuclear Polarization with Single Electron Spins}},
url = {http://link.aps.org/doi/10.1103/PhysRevLett.100.067601},
volume = {100},
year = {2008}
}
@article{Koppens2005,
abstract = {We observed mixing between two-electron singlet and triplet states in a double quantum dot, caused by interactions with nuclear spins in the host semiconductor. This mixing was suppressed when we applied a small magnetic field or increased the interdot tunnel coupling and thereby the singlet-triplet splitting. Electron transport involving transitions between triplets and singlets in turn polarized the nuclei, resulting in marked bistabilities. We extract from the fluctuating nuclear field a limitation on the time-averaged spin coherence time T2* of 25 nanoseconds. Control of the electron-nuclear interaction will therefore be crucial for the coherent manipulation of individual electron spins.},
author = {Koppens, F H L and Folk, J a and Elzerman, J M and Hanson, R and van Beveren, L H Willems and Vink, I T and Tranitz, H P and Wegscheider, W and Kouwenhoven, L P and Vandersypen, L M K},
doi = {10.1126/science.1113719},
file = {:C$\backslash$:/Users/Steffi/AppData/Local/Mendeley Ltd./Mendeley Desktop/Downloaded/Koppens et al. - 2005 - Control and detection of singlet-triplet mixing in a random nuclear field.pdf:pdf},
issn = {1095-9203},
journal = {Science},
month = {aug},
number = {5739},
pages = {1346--50},
pmid = {16037418},
title = {{Control and detection of singlet-triplet mixing in a random nuclear field.}},
url = {http://www.ncbi.nlm.nih.gov/pubmed/16037418},
volume = {309},
year = {2005}
}
@article{Johnson2005,
abstract = {The spin of a confined electron, when oriented originally in some direction, will lose memory of that orientation after some time. Physical mechanisms leading to this relaxation of spin memory typically involve either coupling of the electron spin to its orbital motion or to nuclear spins. Relaxation of confined electron spin has been previously measured only for Zeeman or exchange split spin states, where spin-orbit effects dominate relaxation; spin flips due to nuclei have been observed in optical spectroscopy studies. Using an isolated GaAs double quantum dot defined by electrostatic gates and direct time domain measurements, we investigate in detail spin relaxation for arbitrary splitting of spin states. Here we show that electron spin flips are dominated by nuclear interactions and are slowed by several orders of magnitude when a magnetic field of a few millitesla is applied. These results have significant implications for spin-based information processing.},
author = {Johnson, a C and Petta, J R and Taylor, J M and Yacoby, a and Lukin, M D and Marcus, C M and Hanson, M P and Gossard, a C},
doi = {10.1038/nature03815},
file = {:C$\backslash$:/Users/Steffi/AppData/Local/Mendeley Ltd./Mendeley Desktop/Downloaded/Johnson et al. - 2005 - Triplet-singlet spin relaxation via nuclei in a double quantum dot.pdf:pdf},
issn = {1476-4687},
journal = {Nature},
month = {jun},
number = {7044},
pages = {925--8},
pmid = {15944715},
title = {{Triplet-singlet spin relaxation via nuclei in a double quantum dot.}},
url = {http://www.ncbi.nlm.nih.gov/pubmed/15944715},
volume = {435},
year = {2005}
}
@article{Zumbuhl2002,
author = {Zumb{\"{u}}hl, D. and Miller, J. and Marcus, C. and Campman, K. and Gossard, a.},
doi = {10.1103/PhysRevLett.89.276803},
file = {:C$\backslash$:/Users/Steffi/AppData/Local/Mendeley Ltd./Mendeley Desktop/Downloaded/Zumb{\"{u}}hl et al. - 2002 - Spin-Orbit Coupling, Antilocalization, and Parallel Magnetic Fields in Quantum Dots.pdf:pdf},
issn = {0031-9007},
journal = {Physical Review Letters},
month = {dec},
number = {27},
pages = {276803},
title = {{Spin-Orbit Coupling, Antilocalization, and Parallel Magnetic Fields in Quantum Dots}},
url = {http://link.aps.org/doi/10.1103/PhysRevLett.89.276803},
volume = {89},
year = {2002}
}
@article{Foletti2009,
author = {Foletti, Sandra and Bluhm, Hendrik and Mahalu, Diana and Umansky, Vladimir and Yacoby, Amir},
doi = {10.1038/nphys1424},
file = {:C$\backslash$:/Users/Steffi/AppData/Local/Mendeley Ltd./Mendeley Desktop/Downloaded/Foletti et al. - 2009 - Universal quantum control of two-electron spin quantum bits using dynamic nuclear polarization.pdf:pdf},
issn = {1745-2473},
journal = {Nature Physics},
month = {oct},
number = {12},
pages = {903--908},
publisher = {Nature Publishing Group},
title = {{Universal quantum control of two-electron spin quantum bits using dynamic nuclear polarization}},
url = {http://www.nature.com/doifinder/10.1038/nphys1424},
volume = {5},
year = {2009}
}
@article{Pioro-Ladriere2008,
author = {Pioro-Ladri{\`{e}}re, M. and Obata, T. and Tokura, Y. and Shin, Y.-S. and Kubo, T. and Yoshida, K. and Taniyama, T. and Tarucha, S.},
doi = {10.1038/nphys1053},
file = {:C$\backslash$:/Users/Steffi/AppData/Local/Mendeley Ltd./Mendeley Desktop/Downloaded/Pioro-Ladri{\`{e}}re et al. - 2008 - Electrically driven single-electron spin resonance in a slanting Zeeman field.pdf:pdf},
issn = {1745-2473},
journal = {Nature Physics},
month = {aug},
number = {10},
pages = {776--779},
title = {{Electrically driven single-electron spin resonance in a slanting Zeeman field}},
url = {http://www.nature.com/doifinder/10.1038/nphys1053},
volume = {4},
year = {2008}
}
@article{Tokura2006,
author = {Tokura, Yasuhiro and van der Wiel, Wilfred and Obata, Toshiaki and Tarucha, Seigo},
doi = {10.1103/PhysRevLett.96.047202},
file = {:C$\backslash$:/Users/Steffi/AppData/Local/Mendeley Ltd./Mendeley Desktop/Downloaded/Tokura et al. - 2006 - Coherent Single Electron Spin Control in a Slanting Zeeman Field.pdf:pdf},
issn = {0031-9007},
journal = {Physical Review Letters},
month = {jan},
number = {4},
pages = {047202},
title = {{Coherent Single Electron Spin Control in a Slanting Zeeman Field}},
url = {http://link.aps.org/doi/10.1103/PhysRevLett.96.047202},
volume = {96},
year = {2006}
}
@article{Nowack2007,
abstract = {Manipulation of single spins is essential for spin-based quantum information processing. Electrical control instead of magnetic control is particularly appealing for this purpose, because electric fields are easy to generate locally on-chip. We experimentally realized coherent control of a single-electron spin in a quantum dot using an oscillating electric field generated by a local gate. The electric field induced coherent transitions (Rabi oscillations) between spin-up and spin-down with 90 degrees rotations as fast as approximately 55 nanoseconds. Our analysis indicated that the electrically induced spin transitions were mediated by the spin-orbit interaction. Taken together with the recently demonstrated coherent exchange of two neighboring spins, our results establish the feasibility of fully electrical manipulation of spin qubits.},
author = {Nowack, K C and Koppens, F H L and Nazarov, Yu V and Vandersypen, L M K},
doi = {10.1126/science.1148092},
file = {:C$\backslash$:/Users/Steffi/AppData/Local/Mendeley Ltd./Mendeley Desktop/Downloaded/Nowack et al. - 2007 - Coherent control of a single electron spin with electric fields.pdf:pdf},
issn = {1095-9203},
journal = {Science (New York, N.Y.)},
month = {nov},
number = {5855},
pages = {1430--3},
pmid = {17975030},
title = {{Coherent control of a single electron spin with electric fields.}},
url = {http://www.ncbi.nlm.nih.gov/pubmed/17975030},
volume = {318},
year = {2007}
}
@article{Lai2006,
author = {Lai, C. and Maletinsky, P. and Badolato, a. and Imamoglu, a.},
doi = {10.1103/PhysRevLett.96.167403},
file = {:C$\backslash$:/Users/Steffi/AppData/Local/Mendeley Ltd./Mendeley Desktop/Downloaded/Lai et al. - 2006 - Knight-Field-Enabled Nuclear Spin Polarization in Single Quantum Dots.pdf:pdf},
issn = {0031-9007},
journal = {Physical Review Letters},
month = {apr},
number = {16},
pages = {167403},
title = {{Knight-Field-Enabled Nuclear Spin Polarization in Single Quantum Dots}},
url = {http://link.aps.org/doi/10.1103/PhysRevLett.96.167403},
volume = {96},
year = {2006}
}
@article{Shulman1958,
author = {{Shulman, R.,Wylunda, B.J. , Hrostowski}, H.J.},
file = {:C$\backslash$:/Users/Steffi/AppData/Local/Mendeley Ltd./Mendeley Desktop/Downloaded/Shulman, R.,Wylunda, B.J. , Hrostowski - 1958 - Nuclear Magnetic Resonance.pdf:pdf},
number = {1955},
pages = {808--809},
title = {{Nuclear Magnetic Resonance}},
volume = {109},
year = {1958}
}
@article{Paget1977,
author = {Paget, D.},
file = {:C$\backslash$:/Users/Steffi/AppData/Local/Mendeley Ltd./Mendeley Desktop/Downloaded/Paget - 1977 - Low field electron-nuclear spin coupling in gallium arsenide under optical pumping conditions.pdf:pdf},
journal = {Physical Review B},
number = {12},
pages = {5780--5796},
title = {{Low field electron-nuclear spin coupling in gallium arsenide under optical pumping conditions}},
volume = {15},
year = {1977}
}
@article{Meiboom1958,
author = {Meiboom, S. and Gill, D.},
doi = {10.1063/1.1716296},
file = {:C$\backslash$:/Users/Steffi/AppData/Local/Mendeley Ltd./Mendeley Desktop/Downloaded/Meiboom, Gill - 1958 - Modified Spin-Echo Method for Measuring Nuclear Relaxation Times.pdf:pdf},
issn = {00346748},
journal = {Review of Scientific Instruments},
number = {8},
pages = {688},
title = {{Modified Spin-Echo Method for Measuring Nuclear Relaxation Times}},
url = {http://scitation.aip.org/content/aip/journal/rsi/29/8/10.1063/1.1716296},
volume = {29},
year = {1958}
}
@phdthesis{Foletti2010,
author = {Foletti, Sandra Elisabetta},
file = {:C$\backslash$:/Users/Steffi/AppData/Local/Mendeley Ltd./Mendeley Desktop/Downloaded/Foletti - 2010 - Manipulation and coherence of a two-electron logical spin qubit using GaAs double quantum dots.pdf:pdf},
number = {April},
title = {{Manipulation and coherence of a two-electron logical spin qubit using GaAs double quantum dots}},
year = {2010}
}
@article{Shaji2008,
author = {Shaji, Nakul and Simmons, C. B. and Thalakulam, Madhu and Klein, Levente J. and Qin, Hua and Luo, H. and Savage, D. E. and Lagally, M. G. and Rimberg, a. J. and Joynt, R. and Friesen, M. and Blick, R. H. and Coppersmith, S. N. and Eriksson, M. a.},
doi = {10.1038/nphys988},
file = {:C$\backslash$:/Users/Steffi/AppData/Local/Mendeley Ltd./Mendeley Desktop/Downloaded/Shaji et al. - 2008 - Spin blockade and lifetime-enhanced transport in a few-electron SiSiGe double quantum dot.pdf:pdf},
issn = {1745-2473},
journal = {Nature Physics},
month = {jun},
number = {7},
pages = {540--544},
title = {{Spin blockade and lifetime-enhanced transport in a few-electron Si/SiGe double quantum dot}},
url = {http://www.nature.com/doifinder/10.1038/nphys988},
volume = {4},
year = {2008}
}
@article{DiVincenzo2000,
abstract = {After a brief introduction to the principles and promise of quantum information processing, the requirements for the physical implementation of quantum computation are discussed. These five requirements, plus two relating to the communication of quantum information, are extensively explored and related to the many schemes in atomic physics, quantum optics, nuclear and electron magnetic resonance spectroscopy, superconducting electronics, and quantum-dot physics, for achieving quantum computing.},
archivePrefix = {arXiv},
arxivId = {quant-ph/0002077},
author = {DiVincenzo, David P. and Ibm},
eprint = {0002077},
file = {:C$\backslash$:/Users/Steffi/AppData/Local/Mendeley Ltd./Mendeley Desktop/Downloaded/DiVincenzo, Ibm - 2000 - The Physical Implementation of Quantum Computation.pdf:pdf},
month = {feb},
primaryClass = {quant-ph},
title = {{The Physical Implementation of Quantum Computation}},
url = {http://arxiv.org/abs/quant-ph/0002077},
year = {2000}
}
@article{Sallen2014,
abstract = {Optical and electrical control of the nuclear spin system allows enhancing the sensitivity of NMR applications and spin-based information storage and processing. Dynamic nuclear polarization in semiconductors is commonly achieved in the presence of a stabilizing external magnetic field. Here we report efficient optical pumping of nuclear spins at zero magnetic field in strain-free GaAs quantum dots. The strong interaction of a single, optically injected electron spin with the nuclear spins acts as a stabilizing, effective magnetic field (Knight field) on the nuclei. We optically tune the Knight field amplitude and direction. In combination with a small transverse magnetic field, we are able to control the longitudinal and transverse components of the nuclear spin polarization in the absence of lattice strain--that is, in dots with strongly reduced static nuclear quadrupole effects, as reproduced by our model calculations.},
author = {Sallen, G and Kunz, S and Amand, T and Bouet, L and Kuroda, T and Mano, T and Paget, D and Krebs, O and Marie, X and Sakoda, K and Urbaszek, B},
doi = {10.1038/ncomms4268},
file = {:C$\backslash$:/Users/Steffi/AppData/Local/Mendeley Ltd./Mendeley Desktop/Downloaded/Sallen et al. - 2014 - Nuclear magnetization in gallium arsenide quantum dots at zero magnetic field.pdf:pdf},
issn = {2041-1723},
journal = {Nature communications},
month = {jan},
pages = {3268},
pmid = {24500329},
title = {{Nuclear magnetization in gallium arsenide quantum dots at zero magnetic field.}},
url = {http://www.pubmedcentral.nih.gov/articlerender.fcgi?artid=3926008{\&}tool=pmcentrez{\&}rendertype=abstract},
volume = {5},
year = {2014}
}
@article{Kloeffel2012,
abstract = {Experimental and theoretical progress toward quantum computation with spins in quantum dots (QDs) is reviewed, with particular focus on QDs formed in GaAs heterostructures, on nanowire-based QDs, and on self-assembled QDs. We report on a remarkable evolution of the field where decoherence, one of the main challenges for realizing quantum computers, no longer seems to be the stumbling block it had originally been considered. General concepts, relevant quantities, and basic requirements for spin-based quantum computing are explained; opportunities and challenges of spin-orbit interaction and nuclear spins are reviewed. We discuss recent achievements, present current theoretical proposals, and make several suggestions for further experiments.},
archivePrefix = {arXiv},
arxivId = {1204.5917},
author = {Kloeffel, Christoph and Loss, Daniel},
doi = {10.1146/annurev-conmatphys-030212-184248},
eprint = {1204.5917},
file = {:C$\backslash$:/Users/Steffi/AppData/Local/Mendeley Ltd./Mendeley Desktop/Downloaded/Kloeffel, Loss - 2012 - Prospects for Spin-Based Quantum Computing.pdf:pdf},
keywords = {decoherence,nuclear spins,quantum computer,quantum dot,spin qubit,spin-orbit interaction},
month = {apr},
pages = {21},
title = {{Prospects for Spin-Based Quantum Computing}},
url = {http://arxiv.org/abs/1204.5917},
year = {2012}
}
@article{Rashba2008,
abstract = {Very recently, the electric dipole spin resonance (EDSR) of single electrons in quantum dots was discovered by three independent experimental groups. Remarkably, these observations revealed three different mechanisms of EDSR: coupling of electron spin to its momentum (spin-orbit), to the operator of its position (inhomogeneous Zeeman coupling), and to the hyperfine Overhauser field of nuclear spins. In this paper, I present a unified microscopic theory of these resonances in quantum dots. A mean field theory, derived for all three mechanisms and based on retaining only two-spin correlators, justifies applying macroscopic description of nuclear polarization to the EDSR theory. In the framework of the mean field theory, a fundamental difference in the time dependence of EDSR inherent of these mechanisms is revealed; it changes from the Rabi-type oscillations to a nearly monotonic growth. The theory provides a regular procedure to account for the higher nuclear-spin correlators that become of importance for a wider time span and can change the asymptotic behavior of EDSR. It also allows revealing the effect of electron spin dynamics on the effective coupling between nuclear spins.},
archivePrefix = {arXiv},
arxivId = {0807.2624},
author = {Rashba, Emmanuel I.},
doi = {10.1103/PhysRevB.78.195302},
eprint = {0807.2624},
file = {:C$\backslash$:/Users/Steffi/AppData/Local/Mendeley Ltd./Mendeley Desktop/Downloaded/Rashba - 2008 - Theory of electric dipole spin resonance in quantum dots Mean field theory with Gaussian fluctuations and beyond(2).pdf:pdf},
month = {jul},
number = {i},
pages = {15},
title = {{Theory of electric dipole spin resonance in quantum dots: Mean field theory with Gaussian fluctuations and beyond}},
url = {http://arxiv.org/abs/0807.2624},
year = {2008}
}
@article{Klauser2006,
author = {Klauser, D. and Coish, W. and Loss, Daniel},
doi = {10.1103/PhysRevB.73.205302},
file = {:C$\backslash$:/Users/Steffi/AppData/Local/Mendeley Ltd./Mendeley Desktop/Downloaded/Klauser, Coish, Loss - 2006 - Nuclear spin state narrowing via gate-controlled Rabi oscillations in a double quantum dot.pdf:pdf},
issn = {1098-0121},
journal = {Physical Review B},
month = {may},
number = {20},
pages = {205302},
title = {{Nuclear spin state narrowing via gate-controlled Rabi oscillations in a double quantum dot}},
url = {http://link.aps.org/doi/10.1103/PhysRevB.73.205302},
volume = {73},
year = {2006}
}
@article{Latta2009,
author = {Latta, C. and H{\"{o}}gele, a. and Zhao, Y. and Vamivakas, a. N. and Maletinsky, P. and Kroner, M. and Dreiser, J. and Carusotto, I. and Badolato, a. and Schuh, D. and Wegscheider, W. and Atature, M. and Imamoglu, a.},
doi = {10.1038/nphys1363},
file = {:C$\backslash$:/Users/Steffi/AppData/Local/Mendeley Ltd./Mendeley Desktop/Downloaded/Latta et al. - 2009 - Confluence of resonant laser excitation and bidirectional quantum-dot nuclear-spin polarization.pdf:pdf},
issn = {1745-2473},
journal = {Nature Physics},
month = {aug},
number = {10},
pages = {758--763},
publisher = {Nature Publishing Group},
title = {{Confluence of resonant laser excitation and bidirectional quantum-dot nuclear-spin polarization}},
url = {http://www.nature.com/doifinder/10.1038/nphys1363},
volume = {5},
year = {2009}
}
@article{Greilich2007,
abstract = {The hyperfine interaction of an electron with the nuclei is considered as the primary obstacle to coherent control of the electron spin in semiconductor quantum dots. We show, however, that the nuclei in singly charged quantum dots act constructively by focusing the electron spin precession about a magnetic field into well-defined modes synchronized with a laser pulse protocol. In a dot with a synchronized electron, the light-stimulated fluctuations of the hyperfine nuclear field acting on the electron are suppressed. The information about electron spin precession is imprinted in the nuclei and thereby can be stored for tens of minutes in darkness. The frequency focusing drives an electron spin ensemble into dephasing-free subspaces with the potential to realize single frequency precession of the entire ensemble.},
author = {Greilich, a and Shabaev, a and Yakovlev, D R and Efros, Al L and Yugova, I a and Reuter, D and Wieck, a D and Bayer, M},
doi = {10.1126/science.1146850},
file = {:C$\backslash$:/Users/Steffi/AppData/Local/Mendeley Ltd./Mendeley Desktop/Downloaded/Greilich et al. - 2007 - Nuclei-induced frequency focusing of electron spin coherence.pdf:pdf},
issn = {1095-9203},
journal = {Science (New York, N.Y.)},
month = {sep},
number = {5846},
pages = {1896--9},
pmid = {17901328},
title = {{Nuclei-induced frequency focusing of electron spin coherence.}},
url = {http://www.ncbi.nlm.nih.gov/pubmed/17901328},
volume = {317},
year = {2007}
}
@article{Ladd2010,
author = {Ladd, Thaddeus D. and Press, David and {De Greve}, Kristiaan and McMahon, Peter L. and Friess, Benedikt and Schneider, Christian and Kamp, Martin and H{\"{o}}fling, Sven and Forchel, Alfred and Yamamoto, Yoshihisa},
doi = {10.1103/PhysRevLett.105.107401},
file = {:C$\backslash$:/Users/Steffi/AppData/Local/Mendeley Ltd./Mendeley Desktop/Downloaded/Ladd et al. - 2010 - Pulsed Nuclear Pumping and Spin Diffusion in a Single Charged Quantum Dot.pdf:pdf},
issn = {0031-9007},
journal = {Physical Review Letters},
month = {sep},
number = {10},
pages = {107401},
title = {{Pulsed Nuclear Pumping and Spin Diffusion in a Single Charged Quantum Dot}},
url = {http://link.aps.org/doi/10.1103/PhysRevLett.105.107401},
volume = {105},
year = {2010}
}
@article{Almaden2001,
author = {Vandersypen, Lieven M K and Steffen, Matthias and Breyta, Gregory and Yannoni, Costantino S and Sherwood, Mark H and Chuang, Isaac L},
file = {:C$\backslash$:/Users/Steffi/AppData/Local/Mendeley Ltd./Mendeley Desktop/Downloaded/Vandersypen et al. - 2001 - Experimental realization of Shor ' s quantum factoring algorithm using nuclear magnetic resonance.pdf:pdf},
number = {1976},
pages = {883--887},
title = {{Experimental realization of Shor ' s quantum factoring algorithm using nuclear magnetic resonance}},
volume = {120},
year = {2001}
}
@article{Grover1996,
author = {Grover, Lov K},
file = {:C$\backslash$:/Users/Steffi/AppData/Local/Mendeley Ltd./Mendeley Desktop/Downloaded/Grover - 1996 - A fast quantum mechanical algorithm for database search.pdf:pdf},
isbn = {0897917855},
journal = {Proceedings of the 28th annual ACM sysmposium on Theory of computing},
pages = {212--219},
title = {{A fast quantum mechanical algorithm for database search}},
year = {1996}
}
@article{Peter1994,
author = {Shor, Peter},
file = {:C$\backslash$:/Users/Steffi/AppData/Local/Mendeley Ltd./Mendeley Desktop/Downloaded/Shor - 1994 - Algorithms for Quantum Computation Discrete Logarithms and Factoring.pdf:pdf},
journal = {Foundations of Computer Science},
pages = {124--134},
title = {{Algorithms for Quantum Computation : Discrete Logarithms and Factoring}},
year = {1994}
}
@article{Rubbert2011,
author = {Rubbert, Sebastian},
file = {:C$\backslash$:/Users/Steffi/AppData/Local/Mendeley Ltd./Mendeley Desktop/Downloaded/Rubbert - 2011 - Performance of feedback schemes to suppress nuclear spin uctuations in quantum dots by.pdf:pdf},
number = {September},
title = {{Performance of feedback schemes to suppress nuclear spin uctuations in quantum dots by}},
year = {2011}
}
@article{Reilly2010,
archivePrefix = {arXiv},
arxivId = {arXiv:0803.3082v1},
author = {Reilly, D. J. and Taylor, J. M. and Petta, J. R. and Marcus, C. M. and Hanson, M. P. and Gossard, a. C.},
doi = {10.1103/PhysRevLett.104.236802},
eprint = {arXiv:0803.3082v1},
file = {:C$\backslash$:/Users/Steffi/AppData/Local/Mendeley Ltd./Mendeley Desktop/Downloaded/Reilly et al. - 2010 - Exchange Control of Nuclear Spin Diffusion in a Double Quantum Dot.pdf:pdf},
issn = {0031-9007},
journal = {Physical Review Letters},
month = {jun},
number = {23},
pages = {236802},
title = {{Exchange Control of Nuclear Spin Diffusion in a Double Quantum Dot}},
url = {http://link.aps.org/doi/10.1103/PhysRevLett.104.236802},
volume = {104},
year = {2010}
}
@article{Reilly,
archivePrefix = {arXiv},
arxivId = {arXiv:0707.2946v2},
author = {Reilly, D J and Marcus, C M and Hanson, M P and Gossard, A C},
eprint = {arXiv:0707.2946v2},
file = {:C$\backslash$:/Users/Steffi/AppData/Local/Mendeley Ltd./Mendeley Desktop/Downloaded/Reilly et al. - Unknown - Fast Single-Charge Sensing with an rf Quantum Point Contact.pdf:pdf},
pages = {2--5},
title = {{Fast Single-Charge Sensing with an rf Quantum Point Contact}},
volume = {1}
}
@article{Cerletti2005,
author = {Cerletti, Veronica and Coish, W a and Gywat, Oliver and Loss, Daniel},
doi = {10.1088/0957-4484/16/4/R01},
file = {:C$\backslash$:/Users/Steffi/AppData/Local/Mendeley Ltd./Mendeley Desktop/Downloaded/Cerletti et al. - 2005 - Recipes for spin-based quantum computing.pdf:pdf},
issn = {0957-4484},
journal = {Nanotechnology},
month = {apr},
number = {4},
pages = {R27--R49},
title = {{Recipes for spin-based quantum computing}},
url = {http://stacks.iop.org/0957-4484/16/i=4/a=R01?key=crossref.ffa73be2c017295d2d73db0659d6d44d},
volume = {16},
year = {2005}
}
@article{Latta2011,
author = {Latta, Christian and Srivastava, Ajit and Imamoğlu, Atac},
doi = {10.1103/PhysRevLett.107.167401},
file = {:C$\backslash$:/Users/Steffi/AppData/Local/Mendeley Ltd./Mendeley Desktop/Downloaded/Latta, Srivastava, Imamoğlu - 2011 - Hyperfine Interaction-Dominated Dynamics of Nuclear Spins in Self-Assembled InGaAs Quantum Dots.pdf:pdf},
issn = {0031-9007},
journal = {Physical Review Letters},
month = {oct},
number = {16},
pages = {167401},
title = {{Hyperfine Interaction-Dominated Dynamics of Nuclear Spins in Self-Assembled InGaAs Quantum Dots}},
url = {http://link.aps.org/doi/10.1103/PhysRevLett.107.167401},
volume = {107},
year = {2011}
}
@article{Laird2007,
author = {Laird, E. and Barthel, C. and Rashba, E. and Marcus, C. and Hanson, M. and Gossard, a.},
doi = {10.1103/PhysRevLett.99.246601},
file = {:C$\backslash$:/Users/Steffi/AppData/Local/Mendeley Ltd./Mendeley Desktop/Downloaded/Laird et al. - 2007 - Hyperfine-Mediated Gate-Driven Electron Spin Resonance.pdf:pdf},
issn = {0031-9007},
journal = {Physical Review Letters},
month = {dec},
number = {24},
pages = {246601},
title = {{Hyperfine-Mediated Gate-Driven Electron Spin Resonance}},
url = {http://link.aps.org/doi/10.1103/PhysRevLett.99.246601},
volume = {99},
year = {2007}
}
@article{Vink2009,
author = {Vink, Ivo T. and Nowack, Katja C. and Koppens, Frank H. L. and Danon, Jeroen and Nazarov, Yuli V. and Vandersypen, Lieven M. K.},
doi = {10.1038/nphys1366},
file = {:C$\backslash$:/Users/Steffi/AppData/Local/Mendeley Ltd./Mendeley Desktop/Downloaded/Vink et al. - 2009 - Locking electron spins into magnetic resonance by electron–nuclear feedback.pdf:pdf},
issn = {1745-2473},
journal = {Nature Physics},
month = {aug},
number = {10},
pages = {764--768},
publisher = {Nature Publishing Group},
title = {{Locking electron spins into magnetic resonance by electron–nuclear feedback}},
url = {http://www.nature.com/doifinder/10.1038/nphys1366},
volume = {5},
year = {2009}
}
@article{Kammerloher2012,
author = {Kammerloher, Eugen},
file = {:C$\backslash$:/Users/Steffi/AppData/Local/Mendeley Ltd./Mendeley Desktop/Downloaded/Kammerloher - 2012 - Calibration and Software Interface of an IQ-Mixer by.pdf:pdf},
number = {July},
title = {{Calibration and Software Interface of an IQ-Mixer by}},
year = {2012}
}
@misc{Dickel2013,
author = {Dickel, Christian},
file = {:C$\backslash$:/Users/Steffi/AppData/Local/Mendeley Ltd./Mendeley Desktop/Downloaded/Dickel - 2013 - Nuclear Spin Mediated Landau-Zener Transitions in Double Quantum Dots.pdf:pdf},
number = {September},
title = {{Nuclear Spin Mediated Landau-Zener Transitions in Double Quantum Dots}},
year = {2013}
}
@article{Biercuk2014,
archivePrefix = {arXiv},
arxivId = {arXiv:1101.5189v2},
author = {Biercuk, Michael J and Bluhm, Hendrik},
eprint = {arXiv:1101.5189v2},
file = {:C$\backslash$:/Users/Steffi/AppData/Local/Mendeley Ltd./Mendeley Desktop/Downloaded/Biercuk, Bluhm - 2014 - Phenomenological Study of Decoherence in Solid-State Spin Qubits due to Nuclear Spin Diffusion.pdf:pdf},
pages = {1--11},
title = {{Phenomenological Study of Decoherence in Solid-State Spin Qubits due to Nuclear Spin Diffusion}},
year = {2014}
}
@article{Fowler2009,
author = {Fowler, Austin and Stephens, Ashley and Groszkowski, Peter},
doi = {10.1103/PhysRevA.80.052312},
file = {:C$\backslash$:/Users/Steffi/AppData/Local/Mendeley Ltd./Mendeley Desktop/Downloaded/Fowler, Stephens, Groszkowski - 2009 - High-threshold universal quantum computation on the surface code.pdf:pdf},
issn = {1050-2947},
journal = {Physical Review A},
month = {nov},
number = {5},
pages = {52312},
title = {{High-threshold universal quantum computation on the surface code}},
url = {http://link.aps.org/doi/10.1103/PhysRevA.80.052312},
volume = {80},
year = {2009}
}
@article{Shafiei2013,
author = {Shafiei, M and Nowack, K C and Reichl, C and Wegscheider, W and Vandersypen, L M K},
doi = {10.1103/PhysRevLett.110.107601},
file = {:C$\backslash$:/Users/Steffi/AppData/Local/Mendeley Ltd./Mendeley Desktop/Downloaded/Shafiei et al. - 2013 - Resolving Spin-Orbit- and Hyperfine-Mediated Electric Dipole Spin Resonance in a Quantum Dot.pdf:pdf},
issn = {0031-9007},
journal = {Physical Review Letters},
number = {10},
pages = {107601},
title = {{Resolving Spin-Orbit- and Hyperfine-Mediated Electric Dipole Spin Resonance in a Quantum Dot}},
url = {http://link.aps.org/doi/10.1103/PhysRevLett.110.107601},
volume = {110},
year = {2013}
}
@article{Reilly2008,
author = {Reilly, D J and Taylor, J M and Laird, E A and Petta, J R and Marcus, C and Hanson, M P and Gossard, A C},
doi = {10.1103/PhysRevLett.101.236803},
file = {:C$\backslash$:/Users/Steffi/AppData/Local/Mendeley Ltd./Mendeley Desktop/Downloaded/Reilly et al. - 2008 - Measurement of Temporal Correlations of the Overhauser Field in a Double Quantum Dot.pdf:pdf},
issn = {0031-9007},
journal = {Physical Review Letters},
number = {23},
pages = {236803},
title = {{Measurement of Temporal Correlations of the Overhauser Field in a Double Quantum Dot}},
url = {http://link.aps.org/doi/10.1103/PhysRevLett.101.236803},
volume = {101},
year = {2008}
}
@article{Barthel2009,
author = {Barthel, C and Reilly, D and Marcus, C and Hanson, M and Gossard, A},
doi = {10.1103/PhysRevLett.103.160503},
file = {:C$\backslash$:/Users/Steffi/AppData/Local/Mendeley Ltd./Mendeley Desktop/Downloaded/Barthel et al. - 2009 - Rapid Single-Shot Measurement of a Singlet-Triplet Qubit.pdf:pdf},
issn = {0031-9007},
journal = {Physical Review Letters},
keywords = {quasistatic Overhauser field,single shot fidelity},
number = {16},
pages = {160503},
title = {{Rapid Single-Shot Measurement of a Singlet-Triplet Qubit}},
url = {http://link.aps.org/doi/10.1103/PhysRevLett.103.160503},
volume = {103},
year = {2009}
}
@article{Taylor2007,
author = {Taylor, J and Petta, J and Johnson, A and Yacoby, A and Marcus, C and Lukin, M},
doi = {10.1103/PhysRevB.76.035315},
file = {:C$\backslash$:/Users/Steffi/AppData/Local/Mendeley Ltd./Mendeley Desktop/Downloaded/Taylor et al. - 2007 - Relaxation, dephasing, and quantum control of electron spins in double quantum dots.pdf:pdf},
issn = {1098-0121},
journal = {Physical Review B},
number = {3},
pages = {35315},
title = {{Relaxation, dephasing, and quantum control of electron spins in double quantum dots}},
url = {http://link.aps.org/doi/10.1103/PhysRevB.76.035315},
volume = {76},
year = {2007}
}
@article{Urbaszek2013,
author = {Urbaszek, Bernhard and Marie, Xavier and Amand, Thierry and Krebs, Olivier and Voisin, Paul and Maletinsky, Patrick and H{\"{o}}gele, Alexander and Imamoglu, Atac},
doi = {10.1103/RevModPhys.85.79},
file = {:C$\backslash$:/Users/Steffi/AppData/Local/Mendeley Ltd./Mendeley Desktop/Downloaded/Urbaszek et al. - 2013 - Nuclear spin physics in quantum dots An optical investigation.pdf:pdf},
issn = {0034-6861},
journal = {Rev. Mod. Phys.},
number = {1},
pages = {79--133},
title = {{Nuclear spin physics in quantum dots: An optical investigation}},
url = {http://link.aps.org/doi/10.1103/RevModPhys.85.79},
volume = {85},
year = {2013}
}
@article{Bluhm2010a,
author = {Bluhm, Hendrik and Foletti, Sandra and Neder, Izhar and Rudner, Mark and Mahalu, Diana and Umansky, Vladimir and Yacoby, Amir},
doi = {10.1038/nphys1856},
file = {:C$\backslash$:/Users/Steffi/AppData/Local/Mendeley Ltd./Mendeley Desktop/Downloaded/Bluhm et al. - 2010 - Dephasing time of GaAs electron-spin qubits coupled to a nuclear bath exceeding 200 $\mu$s.pdf:pdf},
issn = {1745-2473},
journal = {Nature Physics},
number = {2},
pages = {109--113},
publisher = {Nature Publishing Group},
title = {{Dephasing time of GaAs electron-spin qubits coupled to a nuclear bath exceeding 200 $\mu$s}},
url = {http://www.nature.com/doifinder/10.1038/nphys1856},
volume = {7},
year = {2010}
}
@phdthesis{Taylor2006,
author = {Taylor, Jacob Mason},
file = {:C$\backslash$:/Users/Steffi/AppData/Local/Mendeley Ltd./Mendeley Desktop/Downloaded/Taylor - 2006 - Hyperfine interactions and quantum information processing in quantum dots.pdf:pdf},
number = {August},
title = {{Hyperfine interactions and quantum information processing in quantum dots}},
year = {2006}
}
@article{Barthel2010,
author = {Barthel, C and Kj{\ae}rgaard, M and Medford, J and Stopa, M and Marcus, C and Hanson, M P and Gossard, A C},
doi = {10.1103/PhysRevB.81.161308},
file = {:C$\backslash$:/Users/Steffi/AppData/Local/Mendeley Ltd./Mendeley Desktop/Downloaded/Barthel et al. - 2010 - Fast sensing of double-dot charge arrangement and spin state with a radio-frequency sensor quantum dot.pdf:pdf},
issn = {1098-0121},
journal = {Physical Review B},
month = {apr},
number = {16},
pages = {161308},
title = {{Fast sensing of double-dot charge arrangement and spin state with a radio-frequency sensor quantum dot}},
url = {http://link.aps.org/doi/10.1103/PhysRevB.81.161308},
volume = {81},
year = {2010}
}
@article{Stepanenko2012,
author = {Stepanenko, Dimitrije and Rudner, Mark and Halperin, Bertrand I and Loss, Daniel},
doi = {10.1103/PhysRevB.85.075416},
file = {:C$\backslash$:/Users/Steffi/AppData/Local/Mendeley Ltd./Mendeley Desktop/Downloaded/Stepanenko et al. - 2012 - Singlet-triplet splitting in double quantum dots due to spin-orbit and hyperfine interactions.pdf:pdf},
issn = {1098-0121},
journal = {Physical Review B},
number = {7},
pages = {75416},
title = {{Singlet-triplet splitting in double quantum dots due to spin-orbit and hyperfine interactions}},
url = {http://link.aps.org/doi/10.1103/PhysRevB.85.075416},
volume = {85},
year = {2012}
}
@article{Petta2005,
abstract = {We demonstrated coherent control of a quantum two-level system based on two-electron spin states in a double quantum dot, allowing state preparation, coherent manipulation, and projective readout. These techniques are based on rapid electrical control of the exchange interaction. Separating and later recombining a singlet spin state provided a measurement of the spin dephasing time, T2*, of approximately 10 nanoseconds, limited by hyperfine interactions with the gallium arsenide host nuclei. Rabi oscillations of two-electron spin states were demonstrated, and spin-echo pulse sequences were used to suppress hyperfine-induced dephasing. Using these quantum control techniques, a coherence time for two-electron spin states exceeding 1 microsecond was observed.},
author = {Petta, J R and Johnson, A C and Taylor, J M and Laird, E A and Yacoby, A and Lukin, M D and Marcus, C and Hanson, M P and Gossard, A C},
doi = {10.1126/science.1116955},
file = {:C$\backslash$:/Users/Steffi/AppData/Local/Mendeley Ltd./Mendeley Desktop/Downloaded/Petta et al. - 2005 - Coherent manipulation of coupled electron spins in semiconductor quantum dots.pdf:pdf},
issn = {1095-9203},
journal = {Science},
month = {sep},
number = {5744},
pages = {2180--2184},
pmid = {16141370},
title = {{Coherent manipulation of coupled electron spins in semiconductor quantum dots}},
url = {http://www.ncbi.nlm.nih.gov/pubmed/16141370},
volume = {309},
year = {2005}
}
@article{Feynman1982,
author = {Feynman, Richard P},
doi = {10.1007/BF02650179},
file = {:C$\backslash$:/Users/Steffi/AppData/Local/Mendeley Ltd./Mendeley Desktop/Downloaded/Feynman - 1982 - Simulating physics with computers.pdf:pdf},
issn = {0020-7748},
journal = {International Journal of Theoretical Physics},
number = {6-7},
pages = {467--488},
title = {{Simulating physics with computers}},
url = {http://link.springer.com/10.1007/BF02650179},
volume = {21},
year = {1982}
}
@article{Bluhm2010,
author = {Bluhm, Hendrik and Foletti, Sandra and Mahalu, Diana and Umansky, Vladimir and Yacoby, Amir},
doi = {10.1103/PhysRevLett.105.216803},
file = {:C$\backslash$:/Users/Steffi/AppData/Local/Mendeley Ltd./Mendeley Desktop/Downloaded/Bluhm et al. - 2010 - Enhancing the Coherence of a Spin Qubit by Operating it as a Feedback Loop That Controls its Nuclear Spin Bath.pdf:pdf},
issn = {0031-9007},
journal = {Physical Review Letters},
month = {nov},
number = {21},
pages = {216803},
title = {{Enhancing the Coherence of a Spin Qubit by Operating it as a Feedback Loop That Controls its Nuclear Spin Bath}},
url = {http://link.aps.org/doi/10.1103/PhysRevLett.105.216803},
volume = {105},
year = {2010}
}
@phdthesis{Fink2012a,
author = {Fink, Thomas},
file = {:C$\backslash$:/Users/Steffi/AppData/Local/Mendeley Ltd./Mendeley Desktop/Downloaded/Fink - 2012 - Probing classical and quantum-mechanical baths with a qubit.pdf:pdf},
school = {RWTH Aachen University},
title = {{Probing classical and quantum-mechanical baths with a qubit}},
year = {2012}
}
@article{Levy2002,
author = {Levy, Jeremy},
doi = {10.1103/PhysRevLett.89.147902},
file = {:C$\backslash$:/Users/Steffi/AppData/Local/Mendeley Ltd./Mendeley Desktop/Downloaded/Levy - 2002 - Universal Quantum Computation with Spin-12 Pairs and Heisenberg Exchange.pdf:pdf},
issn = {0031-9007},
journal = {Physical Review Letters},
month = {sep},
number = {14},
pages = {147902},
title = {{Universal Quantum Computation with Spin-1/2 Pairs and Heisenberg Exchange}},
url = {http://link.aps.org/doi/10.1103/PhysRevLett.89.147902},
volume = {89},
year = {2002}
}
@article{Loss1998,
abstract = {Through the introduction of a new electron spin transport mechanism, a 2D donor electron spin quantum computer architecture is proposed. This design addresses major technical issues in the original Kane design, including spatial oscillations in the exchange coupling strength and cross-talk in gate control. It is also expected that the introduction of nonlocality in qubit interaction will significantly improve the scaling fault-tolerant threshold over the nearest-neighbor linear array.},
archivePrefix = {arXiv},
arxivId = {1607.07025},
author = {Zajac, D. M. and Hazard, T. M. and Mi, X. and Nielsen, E. and Petta, J. R. and Taylor, J. M. and Engel, H. A. and D{\"{u}}r, W. and Yacoby, A. and Marcus, C. M. and Zoller, P. and Lukin, M. D. and Pica, G. and Lovett, B. W. and Bhatt, R. N. and Schenkel, T. and Lyon, S. A. and Hollenberg, L. C L and Greentree, A. D. and Fowler, A. G. and Wellard, C. J. and O'Gorman, Joe and Nickerson, Naomi H. and Ross, Philipp and Morton, John J. L. and Benjamin, Simon C. and Kane, B E and Loss, Daniel and DiVincenzo, David P},
doi = {10.1038/nphys174},
eprint = {1607.07025},
file = {:Q$\backslash$:/spin-QED/Steffi Stuff/References/Architectures/Taylor2005{\_}GaAsarchitecture{\_}nphys174.pdf:pdf;:Q$\backslash$:/spin-QED/Steffi Stuff/References/Architectures/Pica2015.pdf:pdf;:Q$\backslash$:/spin-QED/Steffi Stuff/References/Architectures/Zajac2016{\_}OneDarray{\_}PhysRevApplied.6.054013.pdf:pdf;:Q$\backslash$:/spin-QED/Steffi Stuff/References/Architectures/Hollenberg2006{\_}PhysRevB.74.045311.pdf:pdf;:Q$\backslash$:/spin-QED/Steffi Stuff/References/Architectures/kane1998.pdf:pdf;:Q$\backslash$:/spin-QED/Steffi Stuff/References/Architectures/LossDivincenzo1998QC{\_}PhysRevA.57.120.pdf:pdf;:Q$\backslash$:/spin-QED/Steffi Stuff/References/Architectures/Gorman2015{\_}1406.5149v3.pdf:pdf},
isbn = {1745-2473},
issn = {2056-6387},
journal = {Nature Physics},
number = {3},
pages = {1--8},
pmid = {23804134},
title = {{Two-dimensional architectures for donor-based quantum computing}},
url = {http://link.aps.org/doi/10.1103/PhysRevA.57.120 http://www.nature.com/doifinder/10.1038/30156{\%}5Cnpapers2://publication/doi/10.1038/30156 http://arxiv.org/abs/1406.5149{\%}0Ahttp://dx.doi.org/10.1038/npjqi.2015.19},
volume = {6},
year = {2016}
}
@article{Saeedi2013,
abstract = {We report the strong coupling of a single electron spin and a single microwave photon. The electron spin is trapped in a silicon double quantum dot and the microwave photon is stored in an on-chip high-impedance superconducting resonator. The electric field component of the cavity photon couples directly to the charge dipole of the electron in the double dot, and indirectly to the electron spin, through a strong local magnetic field gradient from a nearby micromagnet. This result opens the way to the realization of large networks of quantum dot based spin qubit registers, removing a major roadblock to scalable quantum computing with spin qubits.},
archivePrefix = {arXiv},
arxivId = {arXiv:1210.0505v1},
author = {Zwanenburg, Floris A. and Dzurak, Andrew S. and Morello, Andrea and Simmons, Michelle Y. and Hollenberg, Lloyd C.L. L. and Klimeck, Gerhard and Rogge, Sven and Coppersmith, Susan N. and Eriksson, Mark A. and Yin, Chunming and Rancic, Milos and {De Boo}, Gabriele G. and Stavrias, Nikolas and McCallum, Jeffrey C. and Sellars, Matthew J. and Rogge, Sven and Weber, Bent and Tan, Y. H.Matthias and Mahapatra, Suddhasatta and Watson, Thomas F. and Ryu, Hoon and Rahman, Rajib and Hollenberg, Lloyd C.L. L. and Klimeck, Gerhard and Simmons, Michelle Y. and Wang, Yu E. and Tankasala, Archana and Hollenberg, Lloyd C.L. L. and Klimeck, Gerhard and Simmons, Michelle Y. and Rahman, Rajib and Trifunovic, Luka and Pedrocchi, Fabio L. and Loss, Daniel and Saraiva, Andr{\'{e}} L. and Baena, A. and Calder{\'{O}}n, Mar{\'{i}}a J. and Koiller, Belita and Samkharadze, N. and Zheng, G. and Kalhor, N. and Brousse, D. and Sammak, A. and Mendes, U. C. and Blais, A. and Scappucci, G. and Vandersypen, L. M.K. K. and Riedel, D. and Fuchs, F. and Kraus, H. and V{\"{a}}th, S. and Sperlich, A. and Dyakonov, V. and Soltamova, A. A. and Baranov, P. G. and Ilyin, V. A. and Astakhov, G. V. and Zajac, D. M. and Hazard, T. M. and Mi, X. and Nielsen, E. and Petta, J. R. and Morton, John J L and McCamey, Dane R. and Eriksson, Mark A. and Lyon, Stephen A. and Lai, N. S. and Lim, W. H. and Yang, C. H. and Zwanenburg, Floris A. and Coish, W. A. and Qassemi, F. and Morello, Andrea and Dzurak, Andrew S. and Kawakami, E. and Scarlino, P. and Ward, D. R. and Braakman, F. R. and Savage, D. E. and Lagally, M. G. and Friesen, Mark and Coppersmith, Susan N. and Eriksson, Mark A. and Vandersypen, L. M.K. K. and Harvey-Collard, Patrick and Jacobson, N. Tobias and Rudolph, Martin and Dominguez, Jason and {Ten Eyck}, Gregory A. and Wendt, Joel R. and Pluym, Tammy and Gamble, John King and Lilly, Michael P. and Pioro-Ladri{\`{e}}re, Michel and Carroll, Malcolm S. and Gonzalez-Zalba, M. F. and Saraiva, Andr{\'{e}} L. and Calder{\'{O}}n, Mar{\'{i}}a J. and Heiss, Dominik and Koiller, Belita and Ferguson, Andrew J. and Boross, P{\'{e}}ter and Sz{\'{e}}chenyi, G{\'{a}}bor and P{\'{a}}lyi, Andr{\'{a}}s and Saeedi, Kamyar and Simmons, Stephanie and Salvail, Jeff Z and Dluhy, Phillip and Riemann, Helge and Abrosimov, Nikolai V and Becker, Peter and Pohl, Hans-Joachim and Morton, John J L and Thewalt, Mike L W and Hanson, R and Petta, J. R. and Tarucha, S and Vandersypen, L. M.K. K.},
doi = {10.1088/0957-4484/27/31/314002},
eprint = {arXiv:1210.0505v1},
file = {:Q$\backslash$:/spin-QED/Steffi Stuff/References/SiQubits/Riedel2012{\_}SiC{\_}PhysRevLett.109.226402.pdf:pdf;:Q$\backslash$:/spin-QED/Steffi Stuff/References/SiQubits/Samkharadze2017{\_}spin-photon coupling.pdf:pdf;:Q$\backslash$:/spin-QED/Steffi Stuff/References/SiQubits/Zwanenburg2013{\_}RevModPhys.85.961{\_}SiQDRev.pdf:pdf;:Q$\backslash$:/spin-QED/Steffi Stuff/References/SiQubits/Theory of one and two donors in Si{\_}1407.8224v1.pdf:pdf;:Q$\backslash$:/spin-QED/Steffi Stuff/References/SiQubits/Trifunovic2013{\_}MagneticCoupling{\_}PhysRevX.3.041023.pdf:pdf;:Q$\backslash$:/spin-QED/Steffi Stuff/References/SiQubits/Yin2013{\_}nature12081.pdf:pdf;:Q$\backslash$:/spin-QED/Steffi Stuff/References/SiQubits/Wang2015{\_}EngineeringExchange{\_}1507.08009v1.pdf:pdf;:Q$\backslash$:/spin-QED/Steffi Stuff/References/SiQubits/Weber2014{\_}Spinblockade and exchange{\_}nnano.2014.63.pdf:pdf;:Q$\backslash$:/spin-QED/Steffi Stuff/References/SiQubits/Petta2016{\_}12qubits.pdf:pdf;:Q$\backslash$:/spin-QED/Steffi Stuff/References/SiQubits/Hanson2007{\_}RevModPhys.79.1217{\_}spinsInFeqElectronQD.pdf:pdf;:Q$\backslash$:/spin-QED/Steffi Stuff/References/SiQubits/Saeedi2013{\_}Science.pdf:pdf;:Q$\backslash$:/spin-QED/Steffi Stuff/References/SiQubits/Morton2011{\_}nature10681.pdf:pdf;:Q$\backslash$:/spin-QED/Steffi Stuff/References/SiQubits/Boross2016{\_}RelaxationDipoleQubit{\_}1602.03691v1.pdf:pdf;:Q$\backslash$:/spin-QED/Steffi Stuff/References/SiQubits/Gonzalez-Zalba2014{\_}Exchangecoupling.pdf:pdf;:Q$\backslash$:/spin-QED/Steffi Stuff/References/SiQubits/Lai2011{\_}PauliBlockInSiDQD.pdf:pdf;:Q$\backslash$:/spin-QED/Steffi Stuff/References/SiQubits/Harvey-Collard2015{\_}nucdrivenElectronSpinRotations{\_}donorToDQD.pdf:pdf;:Q$\backslash$:/spin-QED/Steffi Stuff/References/SiQubits/Kawakami2014{\_}ElectricalControlSiGequbit{\_}nnano.2014.153.pdf:pdf},
isbn = {1530-6984},
issn = {17483395},
journal = {Reviews of Modern Physics},
keywords = {,Quantum computing,central cell correction,donor,donor molecule,dopants,effective mass theory,electron-phonon interaction,electronic materials,exchange coupling,magnetism,phosphorus,quantum chemistry,quantum electronics,quantum information,relaxation,semiconductor,silicon,spin qubits,transistor},
number = {4},
pages = {1--5},
pmid = {25230333},
publisher = {Nature Publishing Group},
title = {{Embracing the quantum limit in silicon computing}},
url = {http://link.aps.org/doi/10.1103/RevModPhys.79.1217 http://www.ncbi.nlm.nih.gov/pubmed/24233718 http://dx.doi.org/10.1038/nnano.2014.153 http://arxiv.org/abs/1607.07025{\%}0Ahttp://dx.doi.org/10.1103/PhysRevApplied.6.054013 http://arxiv.org/abs/1711.02040 http://arxiv.org/abs/1507.08009 http://dx.doi.org/10.1038/nnano.2014.63 http://dx.doi.org/10.1038/nature12081},
volume = {9},
year = {2014}
}
@article{Imamo1999,
abstract = {Graphene has attracted considerable attention in recent years due to its unique physical properties and potential applications. Graphene quantum dots have been proposed as quantum bits, and their excited-state relaxation rates have been studied experimentally. However, their dephasing rates remain unex- plored. In addition, it is still not clear how to implement long-range interaction among qubits for future scalable graphene quantum computing architectures. Here we report a circuit quantum electrodynamics (cQED) experiment using a graphene double quantum dot (DQD) charge qubit and a superconducting reflection-line resonator (RLR). The demonstration of this capacitive coupling between a graphene qubit and a resonator provides a possible approach for me- diating interactions between spatially-separated graphene qubits. Furthermore, taking advantage of sensitive microwave readout measurements using the res- onator, we measure the charge-state dephasing rates in our hybrid graphene nanostructure, which is found to be of the order of GHz. A spectral anal- ysis method is also developed to simultaneously extract: the DQD-resonator coupling strength, the tunneling rate between the DQD charge states, and the charge-state dephasing rate. Our results show that this graphene cQED archi- tecture can be a compelling platform for both graphene physics research and potential applications.},
archivePrefix = {arXiv},
arxivId = {1205.6767},
author = {You, J. Q. and Nori, Franco and Walther, Herbert and Varcoe, Benjamin T H and Englert, Berthold-Georg and Becker, Thomas and Wallraff, Andreas and Schuster, D. I. and Blais, Alexandre and Frunzio, L. and Majer, J. and Devoret, M. H. and Girvin, S. M. and Schoelkopf, R. J. and Viennot, J. J. and Dartiailh, M. C. and Cottet, Audrey and Kontos, Takis and Delbecq, M. R. and Dartiailh, M. C. and Cottet, Audrey and Kontos, Takis and Trif, Mircea and Golovach, Vitaly N. and Loss, Daniel and Togan, E. and Chu, Y. and Trifonov, A. S. and Jiang, L. and Maze, J. and Childress, L. and Dutt, M. V G and S{\o}rensen, A. S. and Hemmer, P. R. and Zibrov, A. S. and Lukin, M. D. and Comber, P G Le and Spear, E and Schmitt, Vivien and Stockklauser, A. and Scarlino, P. and Koski, J. V. and Gasparinetti, S. and Andersen, C. K. and Reichl, C. and Wegscheider, W. and Ihn, T. and Ensslin, K. and Wallraff, Andreas and Sillanp{\"{a}}{\"{a}}, Mika A. and Park, Jae I. and Simmonds, Raymond W. and Sarabi, Bahman and Huang, Peihao and Zimmerman, Neil M. and Samkharadze, N. and Bruno, A. and Scarlino, P. and Zheng, G. and Divincenzo, D. P. and Dicarlo, L. and Vandersypen, L. M K and Raimond, Jean Michel and Brune, Michel and Haroche, Serge and Rauschenbeutel, Arno and Nogues, Gilles and Osnaghi, Stefano and Bertet, Patrice and Brune, Michel and Raimond, Jean Michel and Haroche, Serge and Bennett, Robert and Barlow, Thomas M. and Beige, Almut and Petersson, K. D. and McFaul, L. W. and Schroer, M. D. and Jung, M. and Taylor, J. M. and Houck, A. A. and Petta, J. R. and Palacios-Laloy, Agustin and Petersson, K. D. and Smith, C. G. and Anderson, D. and Atkinson, P. and Jones, G. A C and Ritchie, D. A. and Miller, R. and Northup, T. E. and Birnbaum, K. M. and Boca, A. and Boozer, A. D. and Kimble, H. J. and Mi, X. and Cady, J. V. and Zajac, D. M. and Deelman, P. W. and Petta, J. R. and Stehlik, J. and Edge, L. F. and Petta, J. R. and Majer, J. and Chow, J. M. and Gambetta, J. M. and Koch, Jens and Johnson, B. R. and Schreier, J. A. and Frunzio, L. and Schuster, D. I. and Houck, A. A. and Wallraff, Andreas and Blais, Alexandre and Devoret, M. H. and Girvin, S. M. and Schoelkopf, R. J. and Macha, P. and {Van Der Ploeg}, S. H W and Oelsner, G. and Il'Ichev, E. and Meyer, H. G. and W{\"{u}}nsch, S. and Siegel, M. and Mabuchi, H. and Doherty, A. C. and Kiraz, A. and Michler, P. and Becher, C. and Gayral, B. and Imamoǧlu, A. and Zhang, Lidong and Hu, E. and Schoenfeld, W. V. and Petroff, P. M. and Khitrova, G. and Gibbs, H. M. and Kira, M. and Koch, S. W. and Scherer, A. and Jin, Pei Qing and Marthaler, Michael and Shnirman, Alexander and Sch{\"{o}}n, Gerd and Jaynes, E. T. and Cummings, F. W. and Irish, E. K. and Schwab, K. C. and Hu, Xuedong and Liu, Yu Xi and Nori, Franco and Hornibrook, J. M. and Mitchell, E. E. and Reilly, D. J. and Lewis, C. J. and Reilly, D. J. and Hood, C J and Lynn, T W and Doherty, A. C. and Parkins, A S and Kimble, H. J. and Hao, Yu and Rouxinol, Francisco and Lahaye, M. D. and Gr{\`{e}}zes, C{\'{e}}cile and Saclay, C E A and Goetz, Jan and Deppe, Frank and Haeberlein, Max and Wulschner, Friedrich and Zollitsch, Christoph W. and Meier, Sebastian and Fischer, Michael and Eder, Peter and Xie, Edwar and Fedorov, Kirill G. and Menzel, Edwin P. and Marx, Achim and Gross, Rudolf and G{\"{o}}ppl, M. and Fragner, A. and Baur, M. and Bianchetti, R. and Filipp, S. and Fink, J. M. and Leek, P. J. and Puebla, G. and Steffen, L. and Wallraff, Andreas and Frey, T. and Leek, P. J. and Beck, M. and Blais, Alexandre and Ihn, T. and Ensslin, K. and Wallraff, Andreas and Faraon, Andrei and Santori, Charles and Huang, Zhihong and Acosta, Victor M. and Beausoleil, Raymond G. and Didier, Nicolas and Bourassa, J{\'{e}}r{\^{o}}me and Blais, Alexandre and Dewes, Andreas and Deng, Guang Wei and Wei, Da and Johansson, J. R. and Zhang, Miao Lei and Li, Shu Xiao and Li, Hai Ou and Cao, Gang and Xiao, Ming and Tu, Tao and Guo, Guo Ping Guang Can and Jiang, Hong Wen and Nori, Franco and Guo, Guo Ping Guang Can and Cottet, Audrey and Kontos, Takis and Langford, Nathan K. and Chen, Fei and Sirois, A. J. and Simmonds, Raymond W. and Rimberg, A. J. and Burkard, Guido and Imamoglu, Atac and Burell, Zachary and BRETHEAU and Bothner, D. and Gaber, T. and Kemmler, M. and Koelle, D. and Kleiner, R. and Arnold, Christophe and Demory, Justin and Loo, Vivien and Lema{\^{i}}tre, Aristide and Sagnes, Isabelle and Glazov, Mikha{\"{i}}l and Krebs, Olivier and Voisin, Paul and Senellart, Pascale and Lanco, Lo{\"{i}}c and Armour, A. D. and Blencowe, M. P. and Schwab, K. C. and Wallraff, Andreas and Schuster, D. I. and Blais, Alexandre and Frunzio, L. and Huang, R- S Ren Shou and Majer, J. and Kumar, S and Girvin, S. M. and Schoelkopf, R. J. and Wallraff, Andreas and Girvin, S. M. and Schoelkopf, R. J. and Gao, W B and Fallahi, P and Togan, E. and Miguel-Sanchez, J and Imamoglu, Atac and Imamo, A},
doi = {10.1038/nature11559},
eprint = {1205.6767},
isbn = {0031-9007},
issn = {00319007},
journal = {Physical Review Letters},
keywords = {,Mesoscopics,Microwave coplanar waveguide,Quantum information processing,Quantum physics,Semiconductor quantum dots,Superconducting,Two-level systems,quantum computing,single electron devices,superconducting,two-level systems},
month = {nov},
number = {4},
pages = {1--5},
pmid = {23075988},
publisher = {Nature Publishing Group},
title = {{Circuit quantum electrodynamics with a spin qubit}},
url = {http://www.ncbi.nlm.nih.gov/pubmed/23151586 http://arxiv.org/abs/1210.6512 http://arxiv.org/abs/1310.1897 http://dx.doi.org/10.1016/j.phpro.2012.06.069 https://link.aps.org/doi/10.1103/PhysRevB.68.155311 http://dx.doi.org/10.1038/nature11559 http://arxiv.org/abs/1506.03305 http://arxiv.org/abs/1702.02210 http://dx.doi.org/10.1038/nature09256 http://science.sciencemag.org/content/349/6246/408.abstract http://stacks.iop.org/0034-4885/69/i=5/a=R02?key=crossref.2e5c0e61d04e1066e39543161ec7a7c4 https://link.aps.org/doi/10.1103/PhysRevB.68.064509},
volume = {115},
year = {2012}
}


